\section{Measurement Damage}
\label{sec:damage}

In Step 3, we are cutting off the tails of the Gaussian distribution,
like mice in some gruesome evolution experiment,
to eliminate the shards mentioned in Section \ref{sec:weights}.
 These edge cases will not mix with the rest of
$\mathcal{G}_p$ and may have artificially small size, making them easy
to forge and still pass Step 4.
How do we know this projection won't significantly damage a
valid money state?

First let's define the cut-off interval as the complement of the
``cut-in'' interval:

\begin{displaymath}
\cutoffInterval = \left[2, \Dhalf \right)
\cup \left(\threeDhalf,2D\right]
\end{displaymath}

So the grid diagrams in the cut-off interval are:

\begin{multline*}
\cutoffG
= \\ \{
G: (A(G)=p) \land (d(G) \in
[2, \Dhalf)
\cup (\threeDhalf,2D])
\}
\end{multline*}

Then lets take the norm of the difference between the original valid money state
$\ket{\$_p}$ and the same state after its tails have been cut off,
$\ket{\tilde{\$_p}}$, which ends up just being the sum of the coefficients
squared
of $G \in \mathcal{G}_p$.

\begin{eqnarray*}
|| \ket{\$_p} - \ket{\tilde{\$_p}} || & = &
\sum_{G \in \cutoffG} (\sqrt{q(d(G))})^2 \\
& = & \sum_{d \in \cutoffInterval} \sum_{G \in \mathcal{G}([d])} q(d(G)) \\
& = & \sum_{d \in \cutoffInterval} |\mathcal{G}_p([d])| q(d) \\
& \le & |\mathcal{G}_p|_{\textrm{max}}
\sum_{d \in \cutoffInterval} q(d)
\end{eqnarray*}

In the last inequality, we upper bound using the maximum size of
$\mathcal{G}_p$ for any $d$ in the cut-off interval.

\begin{displaymath}
|\mathcal{G}_p|_{\textrm{max}} =
(\textrm{argmax}_{d \in \cutoffInterval}|\mathcal{G}_p([d])|)
\end{displaymath}

Recall that $q(d)$ is designed to be a Gaussian distribution with
standard deivation $\sqrt{D}$, which we can recenter to zero mean.
Then we
can calculate the area under the distribution $q(d)$ from
$\frac{D}{2}$ to $\frac{3D}{2}$ using the error function:
\footnote{\url{http://en.wikipedia.org/wiki/Normal\_distribution\#Standard\_deviation\_and\_confidence\_intervals}}

\begin{multline*}
F(\mu + n\sigma; \mu, \sigma^2) - F(\mu - n\sigma; \mu, \sigma^2) = 
\Phi(n) - \Phi(-n) \\ = \textrm{erf}(\tfrac{n}{\sqrt{2}}) =
\textrm{erf}(\tfrac{\sqrt{D}}{2\sqrt{2}})
\end{multline*}

Here, $n = \sqrt{D}$, and we can approximate the
error function by:

\begin{displaymath}
\textrm{erf}(\tfrac{\sqrt{D}}{\sqrt{2}}) =
\sqrt{1 - \exp(-\Omega(D^2))} = 1 - \exp(\Omega(-D^2))
\end{displaymath}

Assuming that $|\mathcal{G}_p|_{\textrm{max}}| = \textrm{poly}(D)$,
then we conclude that the damage to a valid money state is exponentially
small.

\begin{displaymath}
|| \ket{\$_p} - \ket{\tilde{\$_p}} || = 1 - \exp(\Omega(-D^2))
\end{displaymath}

%For some intuition about why our assumption
%might be true, consider the following naive upper bound on the 
%For a given size $d$, we have $O(d^2)$ moves which can increase the size
%of the grid to $d+1$, and each of those resulting $d+1$ size grids can
%have $O(d^2)$ moves that maintain the same size. Sum $O(d^4)$ for $d$ in
%some interval linear in $D$ to get a polynomial in
%$D$.