\documentclass[12pt]{article}

\usepackage{fullpage}
\usepackage{fancyhdr}
\usepackage{amsmath}
\pagestyle{fancy}

\rhead{Problem \thesection\\Page \thepage\\Winter 2011}
\lhead{Paul Pham [ppham@cs.washington.edu]\\CSE 599D: Quantum Computing \\Problem Set 4}

%\renewcommand{\headrulewidth}{0.5pt}
\renewcommand{\footrulewidth}{0pt}

\addtolength{\headheight}{42pt} % make space for the rule
\addtolength{\headsep}{6pt} % make space for the rule

\renewcommand{\labelenumi}{\textbf{\alph{enumi})}}
\renewcommand{\labelenumii}{\textbf{\arabic{enumii}}}
%\renewcommand{\thesection}{\small{Problem \arabic{section}.}}
%  \makeatletter
%   \renewcommand{\section}{\@startsection{section}{1}{0mm}
%   {\baselineskip}%
%   {\baselineskip}{\normalfont\normalsize}}%
%   \makeatother
%\renewcommand{\section}{\@startsection{section}{1}}
\def\qopnamewl@#1{\mathop{\fam\z@#1}\nlimits@}
\def\Exp{\mathop{\rm {E}}}
\def\dist{{\rm dist}\,}
\def\mod{{\rm mod}}

\begin{document}
\newcommand{\ket}[1]{|#1 \rangle}
\newcommand{\bra}[1]{\langle #1 |}
\newcommand{\tr}{\text{tr}}
\newcommand{\normtwo}{\frac{1}{\sqrt{2}}}

%%%%%%%%%%%%%%%%%%%%%%%%%%%%%%%%%%%%%%%%%%%%%%%%%%%%%%%%%%%%%%%%%%%%%%%%%%%%%%%
\setcounter{section}{0}
\section{Fidelity and trace distance}

\begin{enumerate}

% Part A
\item
If $\alpha = \ket{\alpha}\bra{\alpha}$ and
$\beta = \ket{\beta}\bra{\beta}$ are the density matrices of pure states,
then we can imagine them as vectors with an angle $\theta$ between them.
Then consider the basis where $\ket{alpha} = \ket{0}$ and
$\ket{beta} = \cos\theta\ket{0} + \sin\theta\ket{1}$, and we have the
following matrix:

\begin{displaymath}
\alpha - \beta = 
\begin{bmatrix}
1 - \cos^2 \theta      & -\cos\theta \sin\theta \\
-\cos\theta \sin\theta & \sin^2 \theta
\end{bmatrix}
\end{displaymath}

Then we have

\begin{displaymath}
T(\alpha, \beta) = \frac{1}{2}||\alpha - \beta||_1 = \tr |A-B|

\end{displaymath}

We can define $\ket{\alpha} = \sum_{i} \sqrt{a_i}\ket{i}$ and
$\ket{\beta} = \sum_{i} \sqrt{b_i}\ket{i}$.
Recall that the fidelity in this case is simply the inner product of
the two vectors. Since each vector has unit magnitude, this is just the
(positive) cosine of the angle between them.

\begin{displaymath}
F(\alpha,\beta) = \sum_{i} \sqrt{a_i b_i} = |\cos\theta|
\end{displaymath}

Finally, we have the following relation between fidelity and trade distance:

\begin{displaymath}
T(\alpha, \beta) = \sqrt{1 - F(\alpha,\beta)^2}
\end{displaymath}

% Part B
\item
We use the fact that trace distance is non-increasing under partial trace.

\begin{displaymath}
T(\alpha, \beta) \ge T(\rho, \sigma) = \frac{1}{2} ||\rho - \sigma||_1
\end{displaymath}

where $\alpha$ and $\beta$ are any purification of $\rho$ and $\sigma$
respectively.

Then from the previous part:
\begin{displaymath}
F(\alpha, \beta)^2 = 1 - T(\alpha, \beta)^2 \le 1 - T(\rho, \sigma)^2 =
\frac{1}{4} || \rho - \sigma ||^2_1
\end{displaymath}

\end{enumerate}

\pagebreak

%%%%%%%%%%%%%%%%%%%%%%%%%%%%%%%%%%%%%%%%%%%%%%%%%%%%%%%%%%%%%%%%%%%%%%%%%%%%%%%
\setcounter{section}{1}
\section{Optimality of super-dense coding and teleportation}

\begin{enumerate}

% Part A
\item
The chance of Alice sending one of $2^n$ $n$-bit messages (chosen uniformly
at random) is $1/2^n$. If $m=n$, then she can send the whole message, and
Bob can guess it with probability $2^{m-n} = 2^0 = 1$.

If $m < n$, then suppose Alice and Bob have some encoding/decoding algorithm

\begin{multline}
E: \{0,1\}^n \rightarrow \{0,1\}^m \\
D: \{0,1\}^m \rightarrow \{0,1\}^n
\end{multline}

By the pigeonhole principle, at least one ciphertext message $y \in \{0,1\}^m$
must be the encoding of more than one plaintext message $x \in \{0,1\}^n$.
In a completely "balanced" encoding, each ciphertext $y$ correspondings to
$2^{n-m}$ plaintext strings, and if Bob receives this $y$, he can do no better
than randomly guess one of these $x$'s with probability $2^{-(n-m)}$. This is the
best case.

In the worst case, the encoding maps more than $2^{n-m}$ $x$'s to a $y$, and
Bob could get one of these. Then his probability of guessing correctly is
less than $2^{-(n-m)}$.

% Part B
\item

% Part C
\item

To teleport each of $n$ qubits, Alice needs to send two cbits to tell Bob
in which basis to measure using the teleportation protocol. If she were able
to do this with fewer than $2n$ cbits, with the EPR pairs that she shares with
Bob being the entangled state, then we would violate the bound in
the parts above.

% Part D
\item 

\end{enumerate}

\pagebreak

%%%%%%%%%%%%%%%%%%%%%%%%%%%%%%%%%%%%%%%%%%%%%%%%%%%%%%%%%%%%%%%%%%%%%%%%%%%%%%%
\section{Partial transpose and data hiding}

\begin{enumerate}

% Part A
\item

% Part B
\item

% Part C
\item

\end{enumerate}

\end{document}
