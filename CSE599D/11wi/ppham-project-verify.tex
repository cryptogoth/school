\section{Verification}

Now we're ready to tackle 
the main steps of verifying whether a pair $(\ket{\phi},p)$ is valid
quantum knot money as follows. Keep in mind that
rejection at any step results in
the whole procedure rejecting.
%, and possibly a visit from Treasury agents.

\begin{enumerate}
\item
Verify that $\ket{\phi}$ is a superposition of valid encoded grid diagrams.
\item
Measure the Alexander polynomial on the state $\ket{\phi}$ and verify
that it is equal to $p$.
At this point, $\ket{\phi}$ is some superposition of
$\mathcal{G}_p$, but may not have the correct weights.
\item
Measure the projector onto grid diagrams with size in the range
$\left[ \Dhalf, \threeDhalf \right]$.
Verify that the outcome is $+1$.
\item
Apply $r$ trials of
Markov chain verification. Verify that all trials have outcome $+1$.
\end{enumerate}

We will now discuss the motivation and important implications of the
last two steps.

%Two important things are worth noting about the last two steps, both of
%which we
%will examine in greater detail: the estimated damage done to a valid money
%state by projecting it to a smaller subspace,
%and how well the mixing properties of the Markov chain allow us
%to distinguish valid from invalid money states.