\section{Verification}

So how do we verify a valid money state $\ket{\$_p}$ made up of possibly
exponentially many, hard-to-find
equivalent grid diagrams? The great insight of this knot scheme is that
the local moves to transform one grid diagram into another can be
compactly represented as a Markov chain. Simply allow a given state to
mix according to this Markov chain for long enough, and it should reach
some stationary distribution (the +1 eigenstates). If we truly started
out with a valid $\ket{\$_p}$, then mixing won't change it. After we
project onto the $+1$ eigenstates, with high probability $\ket{\$_p}$ will
have outcome $+1$.

However, there is one big wrinkle in that we have to limit ourselves
to grid diagrams of a finite size (say $2\overline{D}$), and any moves
which go beyond that limit will not happen in our Markov chain.
There may be equivalent grid diagrams close to that size limit that will
not mix as a result, and therefore these relatively small superpositions
may be easy to forge. To get around this, we can define our superposition
weights (and our corresponding Markov chain moves) to heavily favor
grid diagrams close to some mean (say $\overline{D}$) and where we can
cut off the tails without damaging out states too much. Therefore,
we define the $q_G=q(d(G))$ weights in the previous section to be a
function of the grid diagram's size and to give a
Gaussian distribution centered at $\overline{D}$ with standard deviation
$\sqrt{\overline{D}}$.

The distribution is integer-valued, and the Markov chain walks over
configuration pairs $(G,i)$ of diagram grid $G$ and a label
$i \in [q(d(G))]$. Moves that takes us from $G$ to $G'$ where
$i \ge q(d(G'))$ are not allowed.

So that settles that. Now we're ready to tackle 
the main steps of verifying whether a pair $(\ket{\phi},p)$ is valid
quantum knot money as follows. Keep in mind that
rejection at any step results in
the whole procedure rejecting.

\begin{enumerate}
\item
Verify that $\ket{\phi}$ is a superposition of valid encoded grid diagrams.
\item
Measure the Alexander polynomial on the state $\ket{\phi}$. If it is equal to $p$,
continue. At this point, $\ket{\phi}$ is some superposition of
equivalent grid diagrams, but we must make sure it is the correctly weighted
superposition produced by our minting algorithm.
\item
Measure the projector onto grid diagrams with size in the range
$\left[ \frac{\overline{D}}{2}, \frac{3\overline{D}}{2} \right]$,
essentially cutting off its tails.
If you get outcome $+1$,
continue.
\item
Apply $r$ trials of
Markov chain verification and accept only if all trials have outcome +1.
\end{enumerate}

Two important things are worth noting about the last two steps, both of
which we
will examine in greater detail: the damage done to a valid money state by
cutting off its tails,
and how well the mixing properties of the Markov chain allow us
to distinguish valid from invalid money states.