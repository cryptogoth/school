\section{Knots and Grid Diagrams}

Knots are three-dimensional mathematical objects analogous to a (directed) string
that can be arbitrarily tangled with itself. Knots can be projected into
two-dimensions where every string crossing is drawn as solid (the overcrossing
segment) or broken (the undercrossing segment). However, string is a loose
and messy analogy to deal with. To discretize knots into a useful computational
tool, we can embed them into two-dimensional $d \times d$ grid diagrams,
containing $d$ each of $X$ and $O$ markers, with exactly one of each in every
column and row.

A link is a collection of
one or more directed knots, possibly separable.
If a link is separable, or if a link is the unknot the Alexander polynomial of
its corresponding grid diagrams is 0.

Just as knots are invariant under Reidemeister moves,
the Alexander polynomial is an invariant of knots embedded in grid diagrams
under equivalent moves.
