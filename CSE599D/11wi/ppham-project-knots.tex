\section{Knots}

Knots are like a loop of string which can be arbitrarily tangled
with
itself in three dimensions. We
can represent them (non-uniquely) in two-dimensions with $d \times d$
\emph{grid diagrams}.
%, where strands
%pass vertically and horizontally between $d$ \textsf{X} and
%$d$ \textsf{O} markers,
%one of each kind of marker in each column and row.
%Equivalently, we can encode a grid diagram purely through
%a pair of disjoint permutations $\Pi_{\textsf{X}}$
%and $\Pi_{\textsf{O}}$ on the integers $\{1, \ldots, d\}$.
An Alexander polynomial can be computed for each knot based on its
crossings and is invariant under the Reidemeister moves on all
corresponding gird diagrams.

Based on the leading notation above, you have probably guessed that we
choose our basis $\{G\}$ to consist of all grid diagrams of size
within the arbitrarily chosen range
$[2,2D]$.
Furthermore, we choose $A$ to compute
the Alexander polynomial $p$ of a graph $G$.
We denote the integer size of a grid diagram $G$ by $d(G)$.
Now it is useful to restrict our definition of equivalent basis states
to grid diagrams representing the same underlying knot,
with sizes within a certain interval $[a,b]$ for $a,b \in \mathbb{Z}^+$,
with omission of an interval meaning the
maximum range: $\mathcal{G}_p = \mathcal{G}_p([2,2D])$.

\begin{displaymath}
\mathcal{G}_p([a,b]) = \{G: (A(G) = p) \land (d(G) \in[a,b]) \}
\end{displaymath}

It is conjectured to be hard on average (even for a quantum computer)
to be able to generate all $G \in \mathcal{G}_p([a,b])$ given
uniformly random $p$, $a$, and $b$. We will skip the minting algorithm,
but its one-wayness is based on the random outcome of measuring $p$
out of a superposition of $\{\mathcal{G}_p\}$, together with a classical
private key for signing valid serial numbers.

%This one-wayness, due to quantum measurement, combined with the classical
%one-wayness of digitally signing $p$, result in the one-wayness of
%the minting algorithm. We can then digitally sign $p$ with the bank's
%private key and publish that along with $(\ket{\$_p},p)$.
%However, we can easily create
%$\ket{\$_p}$ for a random $p$ via our minting algorithm:

%\begin{enumerate}
%\item Create a quantum
%superposition of all grid diagrams $\ket{G}$ in a first register.
%\item Compute their Alexander polynomials
%$p$ into a second register.
%\item Measure $p$ to be left with the
%superposition of all $G \in \mathcal{G}_p$.
%corresponding equivalent grid diagrams.
%\end{enumerate}