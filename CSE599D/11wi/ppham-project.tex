\documentclass[twocolumn,10pt]{article}

\usepackage[dvips]{graphicx}
%\usepackage{times}
%\usepackage{fullpage}
\usepackage{eprint}
\usepackage{rotating}
%\usepackage{eepic}
\usepackage{amsfonts}
\usepackage{algorithmic}
\usepackage{amsthm}
\usepackage{amsmath}

\theoremstyle{plain}
\newtheorem{theorem}{Theorem}

\title{Quantum Money From Knots\\
\large{CSE 599D Winter 2011: Final Project}
}
\date{18 March 2011}
\author{Paul Pham}

%%    Q-circuit version 1.06
%    Copyright (C) 2004  Steve Flammia & Bryan Eastin

%    This program is free software; you can redistribute it and/or modify
%    it under the terms of the GNU General Public License as published by
%    the Free Software Foundation; either version 2 of the License, or
%    (at your option) any later version.
%
%    This program is distributed in the hope that it will be useful,
%    but WITHOUT ANY WARRANTY; without even the implied warranty of
%    MERCHANTABILITY or FITNESS FOR A PARTICULAR PURPOSE.  See the
%    GNU General Public License for more details.
%
%    You should have received a copy of the GNU General Public License
%    along with this program; if not, write to the Free Software
%    Foundation, Inc., 59 Temple Place, Suite 330, Boston, MA  02111-1307  USA

\usepackage[matrix,frame,arrow]{xy}
\usepackage{amsmath}
\newcommand{\bra}[1]{\left\langle{#1}\right\vert}
\newcommand{\ket}[1]{\left\vert{#1}\right\rangle}
    % Defines Dirac notation.
\newcommand{\qw}[1][-1]{\ar @{-} [0,#1]}
    % Defines a wire that connects horizontally.  By default it connects to the object on the left of the current object.
    % WARNING: Wire commands must appear after the gate in any given entry.
\newcommand{\qwx}[1][-1]{\ar @{-} [#1,0]}
    % Defines a wire that connects vertically.  By default it connects to the object above the current object.
    % WARNING: Wire commands must appear after the gate in any given entry.
\newcommand{\cw}[1][-1]{\ar @{=} [0,#1]}
    % Defines a classical wire that connects horizontally.  By default it connects to the object on the left of the current object.
    % WARNING: Wire commands must appear after the gate in any given entry.
\newcommand{\cwx}[1][-1]{\ar @{=} [#1,0]}
    % Defines a classical wire that connects vertically.  By default it connects to the object above the current object.
    % WARNING: Wire commands must appear after the gate in any given entry.
\newcommand{\gate}[1]{*{\xy *+<.6em>{#1};p\save+LU;+RU **\dir{-}\restore\save+RU;+RD **\dir{-}\restore\save+RD;+LD **\dir{-}\restore\POS+LD;+LU **\dir{-}\endxy} \qw}
    % Boxes the argument, making a gate.
\newcommand{\meter}{\gate{\xy *!<0em,1.1em>h\cir<1.1em>{ur_dr},!U-<0em,.4em>;p+<.5em,.9em> **h\dir{-} \POS <-.6em,.4em> *{},<.6em,-.4em> *{} \endxy}}
    % Inserts a measurement meter.
\newcommand{\measure}[1]{*+[F-:<.9em>]{#1} \qw}
    % Inserts a measurement bubble with user defined text.
\newcommand{\measuretab}[1]{*{\xy *+<.6em>{#1};p\save+LU;+RU **\dir{-}\restore\save+RU;+RD **\dir{-}\restore\save+RD;+LD **\dir{-}\restore\save+LD;+LC-<.5em,0em> **\dir{-} \restore\POS+LU;+LC-<.5em,0em> **\dir{-} \endxy} \qw}
    % Inserts a measurement tab with user defined text.
\newcommand{\measureD}[1]{*{\xy*+=+<.5em>{\vphantom{#1}}*\cir{r_l};p\save*!R{#1} \restore\save+UC;+UC-<.5em,0em>*!R{\hphantom{#1}}+L **\dir{-} \restore\save+DC;+DC-<.5em,0em>*!R{\hphantom{#1}}+L **\dir{-} \restore\POS+UC-<.5em,0em>*!R{\hphantom{#1}}+L;+DC-<.5em,0em>*!R{\hphantom{#1}}+L **\dir{-} \endxy} \qw}
    % Inserts a D-shaped measurement gate with user defined text.
\newcommand{\multimeasure}[2]{*+<1em,.9em>{\hphantom{#2}} \qw \POS[0,0].[#1,0];p !C *{#2},p \drop\frm<.9em>{-}}
    % Draws a multiple qubit measurement bubble starting at the current position and spanning #1 additional gates below.
    % #2 gives the label for the gate.
    % You must use an argument of the same width as #2 in \ghost for the wires to connect properly on the lower lines.
\newcommand{\multimeasureD}[2]{*+<1em,.9em>{\hphantom{#2}}\save[0,0].[#1,0];p\save !C *{#2},p+LU+<0em,0em>;+RU+<-.8em,0em> **\dir{-}\restore\save +LD;+LU **\dir{-}\restore\save +LD;+RD-<.8em,0em> **\dir{-} \restore\save +RD+<0em,.8em>;+RU-<0em,.8em> **\dir{-} \restore \POS !UR*!UR{\cir<.9em>{r_d}};!DR*!DR{\cir<.9em>{d_l}}\restore \qw}
    % Draws a multiple qubit D-shaped measurement gate starting at the current position and spanning #1 additional gates below.
    % #2 gives the label for the gate.
    % You must use an argument of the same width as #2 in \ghost for the wires to connect properly on the lower lines.
\newcommand{\control}{*-=-{\bullet}}
    % Inserts an unconnected control.
\newcommand{\controlo}{*!<0em,.04em>-<.07em,.11em>{\xy *=<.45em>[o][F]{}\endxy}}
    % Inserts a unconnected control-on-0.
\newcommand{\ctrl}[1]{\control \qwx[#1] \qw}
    % Inserts a control and connects it to the object #1 wires below.
\newcommand{\ctrlo}[1]{\controlo \qwx[#1] \qw}
    % Inserts a control-on-0 and connects it to the object #1 wires below.
\newcommand{\targ}{*{\xy{<0em,0em>*{} \ar @{ - } +<.4em,0em> \ar @{ - } -<.4em,0em> \ar @{ - } +<0em,.4em> \ar @{ - } -<0em,.4em>},*+<.8em>\frm{o}\endxy} \qw}
    % Inserts a CNOT target.
\newcommand{\qswap}{*=<0em>{\times} \qw}
    % Inserts half a swap gate. 
    % Must be connected to the other swap with \qwx.
\newcommand{\multigate}[2]{*+<1em,.9em>{\hphantom{#2}} \qw \POS[0,0].[#1,0];p !C *{#2},p \save+LU;+RU **\dir{-}\restore\save+RU;+RD **\dir{-}\restore\save+RD;+LD **\dir{-}\restore\save+LD;+LU **\dir{-}\restore}
    % Draws a multiple qubit gate starting at the current position and spanning #1 additional gates below.
    % #2 gives the label for the gate.
    % You must use an argument of the same width as #2 in \ghost for the wires to connect properly on the lower lines.
\newcommand{\ghost}[1]{*+<1em,.9em>{\hphantom{#1}} \qw}
    % Leaves space for \multigate on wires other than the one on which \multigate appears.  Without this command wires will cross your gate.
    % #1 should match the second argument in the corresponding \multigate. 
\newcommand{\push}[1]{*{#1}}
    % Inserts #1, overriding the default that causes entries to have zero size.  This command takes the place of a gate.
    % Like a gate, it must precede any wire commands.
    % \push is useful for forcing columns apart.
    % NOTE: It might be useful to know that a gate is about 1.3 times the height of its contents.  I.e. \gate{M} is 1.3em tall.
    % WARNING: \push must appear before any wire commands and may not appear in an entry with a gate or label.
\newcommand{\gategroup}[6]{\POS"#1,#2"."#3,#2"."#1,#4"."#3,#4"!C*+<#5>\frm{#6}}
    % Constructs a box or bracket enclosing the square block spanning rows #1-#3 and columns=#2-#4.
    % The block is given a margin #5/2, so #5 should be a valid length.
    % #6 can take the following arguments -- or . or _\} or ^\} or \{ or \} or _) or ^) or ( or ) where the first two options yield dashed and
    % dotted boxes respectively, and the last eight options yield bottom, top, left, and right braces of the curly or normal variety.
    % \gategroup can appear at the end of any gate entry, but it's good form to pick one of the corner gates.
    % BUG: \gategroup uses the four corner gates to determine the size of the bounding box.  Other gates may stick out of that box.  See \prop. 
\newcommand{\rstick}[1]{*!L!<-.5em,0em>=<0em>{#1}}
    % Centers the left side of #1 in the cell.  Intended for lining up wire labels.  Note that non-gates have default size zero.
\newcommand{\lstick}[1]{*!R!<.5em,0em>=<0em>{#1}}
    % Centers the right side of #1 in the cell.  Intended for lining up wire labels.  Note that non-gates have default size zero.
\newcommand{\ustick}[1]{*!D!<0em,-.5em>=<0em>{#1}}
    % Centers the bottom of #1 in the cell.  Intended for lining up wire labels.  Note that non-gates have default size zero.
\newcommand{\dstick}[1]{*!U!<0em,.5em>=<0em>{#1}}
    % Centers the top of #1 in the cell.  Intended for lining up wire labels.  Note that non-gates have default size zero.
\newcommand{\Qcircuit}{\xymatrix @*=<0em>}
    % Defines \Qcircuit as an \xymatrix with entries of default size 0em.


\begin{document}

\newcommand{\ket}[1]{|#1 \rangle}
\newcommand{\bra}[1]{\langle #1 |}
\newcommand{\braket}[2]{\langle #1|#2 \rangle}
\newcommand{\normtwo}{\frac{1}{\sqrt{2}}}
\newcommand{\norm}[1]{\parallel #1 \parallel}
\newcommand{\Dhalf}{\tfrac{D}{2}}
\newcommand{\threeDhalf}{\tfrac{3D}{2}}
\newcommand{\cutoffInterval}{\overline{\left[\Dhalf, \threeDhalf\right]}}
\newcommand{\cutoffG}{\mathcal{G}_p(\cutoffInterval)}

\maketitle

\section{Abstract}

In this project, I critique the public-key quantum money scheme of
\cite{Farhi2010} based on the hardness of finding equivalent
knots. Making assumptions based on conjectured properties
of grid diagrams,
I estimate the damage to a valid
money state from the verification procedure and characterize the
desired behavior of Markov chain mixing in order to distinguish
valid and invalid money states with a polynomial number of trials.
In conclusion, I propose extending this work for more money-like
features.

\section{Introduction}

General quantum states cannot be cloned. Is it possible to
turn this apparent disadvantage into a cryptographic advantage,
implementing money that is hard to counterfeit?
We want valid money tokens to be easily \emph{minted}
(possibly with some classical secret) by a central
bank but easily \emph{verified}
by anyone (with access to a quantum computer).

This project provides a critical summary of a recently proposed 
 quantum money scheme based on the properties
of knots and grid diagrams \cite{Farhi2010}.
The interested reader is referred to that paper for
a good summary of prior work.
While a promising approach, this scheme's Markov-chain-based
verification algorithm is incomplete.
Therefore our criticism will focus on the ability to
distinguish valid and invalid
quantum money states.

Our report is organized as follows.
First, we briefly review knots and how they are used
in the minting algorithm to create valid money states.
Second and most importantly, we analyze the relative ability of
valid and invalid money states to survive the
verification algorithm, specifically measurement damage and
the
desired mixing properties of the given Markov chain. This section
requires some ``reasonable'' assumptions, which in turn point to
possible future extensions, where we conclude.

%\section{Related Work}

%The unforgeability of quantum money was studied as early as Wiesner
%\cite{}. Although his scheme provides information-theoretic security
%in the sense of relying directly on the laws of physics, it has the
%severe disadvantage of involving the mint in every transaction.
%Ideally, we would like our quantum money to be publicly verifiable, that is,
%without resorting to the trusted authority for every interaction.
%Aaronson proved that public-key quantum money was possible relative to an oracle




\section{Quantum Money}
\label{sec:model}

Consider money tokens which consist of a quantum state $\ket{\$_p}$ in
a fixed basis $\{\ket{G}\}$, where the labels are taken from a set
$\{G\}$,
and a classical serial number $p$.
We will use $G$ and $\ket{G}$ interchangeably.
Our scheme also includes a classical function
$A: \{G\} \rightarrow \{p\}$, which defines equivalence
classes $\mathcal{G}_p$ of labels $G$ which map to the same $p$.

\begin{displaymath}
\mathcal{G}_p = \{G: A(G) = p \}
\end{displaymath}

We define a valid money state $\ket{\$_p}$ as a (possibly weighted)
superposition of states from $\mathcal{G}_p$, normalized by some constant
$N$.

\begin{displaymath}
\ket{\$_p} = \frac{1}{\sqrt{N}} \sum_{G \in \mathcal{G}_p} q_G \ket{G}
\end{displaymath}

All probabilities and security properties of our scheme scale
with respect to a security parameter $D \in \mathbb{Z}^+$.
%We would like a different serial number $p$ to be produced each time
%with probability exponentially close to 1 to prevent forgery through
%repetition
%of the minting algorithm.
For each $p$, or even from each
$G \in \mathcal{G}_p$, it should be difficult to find the rest of
$\mathcal{G}_p$ (which would let us forge
a specific $\ket{\$_p}$). We will see later that it is useful to
shape the weights $q_G$ of the superposition, rather than have a uniform
distribution. So how do we make public-key quantum money from this?


\section{Knots and Grid Diagrams}

Knots are three-dimensional mathematical objects analogous to a (directed) string
that can be arbitrarily tangled with itself. Knots can be projected into
two-dimensions where every string crossing is drawn as solid (the overcrossing
segment) or broken (the undercrossing segment). However, string is a loose
and messy analogy to deal with. To discretize knots into a useful computational
tool, we can embed them into two-dimensional $d \times d$ grid diagrams,
containing $d$ each of $X$ and $O$ markers, with exactly one of each in every
column and row.

A link is a collection of
one or more directed knots, possibly separable.
If a link is separable, or if a link is the unknot the Alexander polynomial of
its corresponding grid diagrams is 0.

Just as knots are invariant under Reidemeister moves,
the Alexander polynomial is an invariant of knots embedded in grid diagrams
under equivalent moves.


\section{Superposition Weights}
\label{sec:weights}

So how do we verify a valid money state $\ket{\$_p}$ made up of possibly
exponentially many, hard-to-find
equivalent grid diagrams? Note that
the local moves to transform one grid diagram into another can be
compactly represented as a Markov chain. Simply allow a given state to
mix according to this Markov chain for long enough. If we started
out with a valid $\ket{\$_p}$, then mixing won't change it because
all $G\in\mathcal{G}_p$ will mix with each other in isolation from the
grid diagrams of another polynomial $p'$. If its
eigenvalue is close to +1, it should project to the
stationary distributions (the +1 eigenstates) with high probability.

However, there is one big wrinkle in that we have to limit ourselves
to a finite size interval,
so any moves which go outside that will not happen in our Markov chain.
There may be small subsets of $\mathcal{G}_p$, which we'll call
\emph{shards},
close to that size limit that will
not mix as a result, which could be
easy to forge but still pass verification.
To get around this, we can define our superposition
weights (and our corresponding Markov chain moves) to heavily favor
grid diagrams close to some mean and then
cut off the tails. Therefore,
we define our weights $q_G=q(d(G))$ in Section \ref{sec:model}
to be
a Gaussian distribution on the grid diagram's size 
centered at $D$ with standard deviation
$\sqrt{D}$.

\begin{displaymath}
q(d) = 
\begin{cases} 
\lceil \tfrac{1}{N} \exp \tfrac{-(d-D)^2}{2D} \rceil & \text{if } 2 \le d \le 2D\\ 
0 & \text{otherwise }
\end{cases} 
\end{displaymath}

The distribution is integer-valued, and the Markov chain walks over
configuration pairs $(G,i)$ of diagram grid $G$ and a label
$i \in [q(d(G))]$. Moves that takes us from $G$ to $G'$ where
$i \ge q(d(G'))$ are not allowed.
Now we show that cutting of the tails doesn't damage our state too much.

\section{Verification}

So how do we verify a valid money state $\ket{\$_p}$ made up of possibly
exponentially many, hard-to-find
equivalent grid diagrams? The great insight of this knot scheme is that
the local moves to transform one grid diagram into another can be
compactly represented as a Markov chain. Simply allow a given state to
mix according to this Markov chain for long enough, and it should reach
some stationary distribution (the +1 eigenstates). If we truly started
out with a valid $\ket{\$_p}$, then mixing won't change it. After we
project onto the $+1$ eigenstates, with high probability $\ket{\$_p}$ will
have outcome $+1$.

However, there is one big wrinkle in that we have to limit ourselves
to grid diagrams of a finite size (say $2\overline{D}$), and any moves
which go beyond that limit will not happen in our Markov chain.
There may be equivalent grid diagrams close to that size limit that will
not mix as a result, and therefore these relatively small superpositions
may be easy to forge. To get around this, we can define our superposition
weights (and our corresponding Markov chain moves) to heavily favor
grid diagrams close to some mean (say $\overline{D}$) and where we can
cut off the tails without damaging out states too much. Therefore,
we define the $q_G=q(d(G))$ weights in the previous section to be a
function of the grid diagram's size and to give a
Gaussian distribution centered at $\overline{D}$ with standard deviation
$\sqrt{\overline{D}}$.

The distribution is integer-valued, and the Markov chain walks over
configuration pairs $(G,i)$ of diagram grid $G$ and a label
$i \in [q(d(G))]$. Moves that takes us from $G$ to $G'$ where
$i \ge q(d(G'))$ are not allowed.

So that settles that. Now we're ready to tackle 
the main steps of verifying whether a pair $(\ket{\phi},p)$ is valid
quantum knot money as follows. Keep in mind that
rejection at any step results in
the whole procedure rejecting.

\begin{enumerate}
\item
Verify that $\ket{\phi}$ is a superposition of valid encoded grid diagrams.
\item
Measure the Alexander polynomial on the state $\ket{\phi}$. If it is equal to $p$,
continue. At this point, $\ket{\phi}$ is some superposition of
equivalent grid diagrams, but we must make sure it is the correctly weighted
superposition produced by our minting algorithm.
\item
Measure the projector onto grid diagrams with size in the range
$\left[ \frac{\overline{D}}{2}, \frac{3\overline{D}}{2} \right]$,
essentially cutting off its tails.
If you get outcome $+1$,
continue.
\item
Apply $r$ trials of
Markov chain verification and accept only if all trials have outcome +1.
\end{enumerate}

Two important things are worth noting about the last two steps, both of
which we
will examine in greater detail: the damage done to a valid money state by
cutting off its tails,
and how well the mixing properties of the Markov chain allow us
to distinguish valid from invalid money states.

\section{Damage from Measuring Valid Money States}

In Step 3, we are cutting off the tails of the Gaussian distribution
to eliminate a certain class of equivalent knot diagrams with size close
to $2\overline{D}$ or $2$ which are easy to create due to the edge cases
in our Markov chain moves. Otherwise, these easily forgeable states would
pass Step 4. How do we know this projection won't significantly damage a valid
money state?

First let's define the set of all equivalent
grid diagrams $G$ with Alexander polynomial $p$ and with dimension outside
the cutoff regions:

\begin{displaymath}
\mathcal{G} = \{
G:A(G)=p \land d(G) \in
[2, \frac{\overline{D}}{2})
\cap (\frac{3\overline{D}}{2},2\overline{D}]
\}
\end{displaymath}

Then lets take the norm of the difference between the original valid money state
$\ket{\psi}$ and the same state after its tails have been cut off,
$\ket{\tilde{\psi}}$, which ends up just being the sum of the coefficients
squared
of grid diagrams in $\mathcal{G}$.

\begin{displaymath}
|| \ket{\psi} - \ket{\tilde{\psi}} || \le
\sum_{G \in \mathcal{G}} (\sqrt{q(d(G))})^2
\end{displaymath}

Recall that $q(d)$ is designed to be a Gaussian distribution with
standard deivation $\sqrt{\overline{D}}/2$, which we can recenter to zero mean.
Then we
can calculate the area under the distribution from
$\frac{\overline{D}}{2}$ to $\frac{3\overline{D}}{2}$ using the error function:
\cite{WikipediaNormal}

\begin{multline*}
F(\mu + n\sigma; \mu, \sigma^2) - F(\mu - n\sigma; \mu, \sigma^2) = 
\Phi(n) - \Phi(-n) \\ = \textrm{erf}(\frac{n}{\sqrt{2}}) =
\textrm{erf}(\frac{\sqrt{\overline{D}}}{2\sqrt{2}})
\end{multline*}

Here, $n = \sqrt{\overline{D}}/2$, and we can approximate the
error function by:

\begin{displaymath}
erf(\frac{\overline{D}}{2\sqrt{2}}) =
\sqrt{1 - \exp(-\Omega(\overline{D}^2))} = 1 - \exp(\Omega(\overline{D}^2))
\end{displaymath}


\section{Markov Chain Mixing}

In Step 4 of the verification scheme, we apply a Markov chain
$\hat{B}$ and then
project onto its +1 eigenstates. This depends on valid money states
$\ket{\$_p}$
having eigenvalues very close to (but not exactly equal to) +1,
which is suggested by the previous
section. But are $\ket{\$_p}$ eigenvalues much closer than
the eigenvalues corresponding to invalid money?
Our only
hope is
%that the eigenvalues for $\ket{\$_p}$ being exponentially
%close to one and
that the ``invalid'' eigenvalues
are at least
polynomially farther away.

Unfortunately, we don't understand enough about knots to make that
claim for this particular Markov chain. This is the biggest open
question and avenue for attack in our knot-based scheme.
In particular, we don't know
the eigenvalue gap, if any, between the lowest eigenvalue of
a $\ket{\$_p}$ (call it $(1-a), a \in [0,1)$) and the highest eigenvalue of any other
eigenstates (call it $(1-b), b \in [0,1)$). We are guaranteed to be exponentially close
to some eigenstate of $\hat{B}$ after calculating and measuring the
Alexander polynomial in Step 2 above.

In our wildest dreams, what would we like to be true about the eigenvalues
of $\hat{B}$?
First off, we'd like $b > a$. Next, we'd like $a$ to be
exponentially small, so that a $\ket{\$_p}$ doesn't degrade under
$r$ repetitions of Markov chain verification and still projects
to a +1 eigenstate with high probability.
Finally, we'd like $b$ to be at least polynomially away from 1, so
that under $r$ repetitions of Markov chain verification, it
projects to a +1 eigenstate with low probability.

%\begin{eqnarray*}
%a & = & \frac{1}{\exp(\Omega(D))} \\
%b & = & \frac{1}{\Omega(D)}
%\end{eqnarray*}

How high is high and how low is low?
We would like to show that the difference in probabilities increases
exponentially close to 1 with $r$, where in the first inequality below
we use the union bound.

\begin{multline*}
(1-a)^r - (1-b)^r \ge (1 - ra) - (1 - rb) \\
= r(b-a) =
r \left(\tfrac{1}{\exp(\Omega(D))} - \tfrac{1}{\Omega(D)}\right)
\end{multline*}

Therefore, if $(b-a)$ also increases exponentially closer to 1, we can
get away with $r = \textrm{poly}(D)$ repetitions, so that our
Markov chain verification procedure is tractable.

\section{Future Extensions}

To extend this scheme or prove it secure, we would need a better
understanding of knots and quantum algorithms for them. The
two obvious future extensions are to come up with a quantum algorithm
for knot equivalence or
to prove an eigenvalue gap for the Markov chain in the verification
scheme. Aside from those, here are a
few other interesting directions:

\begin{enumerate}
\item
For a given security parameter $D$, 
are there sufficiently many Alexander polynomials (serial numbers)
available to meet the global demand for quantum bills?
To do this, we would need to lower-bound $|\{\mathcal{G}_p\}|$,
not including the unknot which has empirically been shown
to
occupy the vast majority of grid diagrams.
\item
Quantum bills currently have no denomination associated with
them and so are of unit value.
Is it possible to associate a denomination with quantum money, or to have
it be dividable or combinable?
\item 
It turns out that a certain class of states can be efficiently copied, which
includes the eigenstates of the addition operator, $\ket{\psi_{n,k}}$,
used to enact arbitrary controlled phase
rotations in the quantum compiling algorithm of Kitaev, Shen, and Vyalyi
\cite{KSV02}.
%The state to copy (say $\ket{\psi_{n,1}}$, which is hard
%to produce), but is of the
%same form as the ``empty'' register to hold the new copy
%($\ket{\psi_{n,0}}$, which
%is easy to produce).
%As in
%\begin{displaymath}
%$\ket{\psi_{n,1}}\otimes\ket{\psi_{n,0}} \rightarrow
%\ket{\psi_{n,1}}\otimes\ket{\psi_{n,1}}$.
%\end{displaymath}
If it turns out that $\ket{\$_p}$ or some part of it is
within that class of states, then we can forge it with non-negligible
probability.
\end{enumerate}

Results which have emerged since the main paper \cite{Farhi2010}
include a new online attack for Wiesner's original scheme
which involves the bank returning bogus bills,
also addressed in \cite{Lutomirski2010} by 
\emph{collision-free} quantum money
protocols which prevent currency inflation.
Furthermore, if deciding knot equivalence in two dimensions is ever
solved, one might hope that lifting knots into
three-dimensional \emph{cube diagrams}
will evade solution for longer \cite{Baldridge2009}.

In conclusion, our hopes are raised by the promise of
this new public-key quantum money scheme, but we also poke at some of
its shortcomings. We've only scratched the surface of this fascinating
topic, leaving out many bad jokes, and
we look forward to future developments in this field.


\bibliography{ppham-project}
\bibliographystyle{tocplain}

\end{document}
