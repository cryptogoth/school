\section{Introduction}

General quantum states cannot be cloned, but this apparent
algorithmic disadvantage can be turned into a cryptographic advantage.
Money is a common implementation of a
real-life, physical one-way function:
we want valid bills and coins to be easily creatable (via a
\emph{minting algorithm} possibly with some classical secret) by a central
bank but easily checkable (by a public \emph{verification algorithm})
by anyone with access to a quantum computer.

This project provides a critical summary of a recent proposed 
 quantum money scheme based on the properties
mathematical knots \cite{Farhi2010}.
The interested reader is referred to that paper for
a good summary of prior work.
While a promising approach, this scheme's Markov-chain-based
verification algorithm is incomplete
and may not be able to distinguish valid and invalid quantum money states.

First, we briefly review knots and how they are used
in the minting algorithm to create valid money states.
Second, we move on to the main part of this paper, the dissection of the
verification algorithm, including a calculation bounding how much
damage is done to a valid money state and a discussion about our
desired mixing properties for the Markov chain part.
Then we expand upon existing attacks for this scheme.
Finally, we conclude with future extensions to this exciting work to
make the scheme more concrete.

%\section{Related Work}

%The unforgeability of quantum money was studied as early as Wiesner
%\cite{}. Although his scheme provides information-theoretic security
%in the sense of relying directly on the laws of physics, it has the
%severe disadvantage of involving the mint in every transaction.
%Ideally, we would like our quantum money to be publicly verifiable, that is,
%without resorting to the trusted authority for every interaction.
%Aaronson proved that public-key quantum money was possible relative to an oracle


