\section{Superposition Weights}
\label{sec:weights}

So how do we verify a valid money state $\ket{\$_p}$ made up of possibly
exponentially many, hard-to-find
equivalent grid diagrams? Note that
the local moves to transform one grid diagram into another can be
compactly represented as a Markov chain. Simply allow a given state to
mix according to this Markov chain for long enough. If we started
out with a valid $\ket{\$_p}$, then mixing won't change it because
all $G\in\mathcal{G}_p$ will mix with each other in isolation from the
grid diagrams of another polynomial $p'$. If its
eigenvalue is close to +1, it should project to the
stationary distributions (the +1 eigenstates) with high probability.

However, there is one big wrinkle in that we have to limit ourselves
to a finite size interval,
so any moves which go outside that will not happen in our Markov chain.
There may be small subsets of $\mathcal{G}_p$, which we'll call
\emph{shards},
close to that size limit that will
not mix as a result, which could be
easy to forge but still pass verification.
To get around this, we can define our superposition
weights (and our corresponding Markov chain moves) to heavily favor
grid diagrams close to some mean and then
cut off the tails. Therefore,
we define our weights $q_G=q(d(G))$ in Section \ref{sec:model}
to be
a Gaussian distribution on the grid diagram's size 
centered at $D$ with standard deviation
$\sqrt{D}$.

\begin{displaymath}
q(d) = 
\begin{cases} 
\lceil \tfrac{1}{N} \exp \tfrac{-(d-D)^2}{2D} \rceil & \text{if } 2 \le d \le 2D\\ 
0 & \text{otherwise }
\end{cases} 
\end{displaymath}

The distribution is integer-valued, and the Markov chain walks over
configuration pairs $(G,i)$ of diagram grid $G$ and a label
$i \in [q(d(G))]$. Moves that takes us from $G$ to $G'$ where
$i \ge q(d(G'))$ are not allowed.
Now we show that cutting of the tails doesn't damage our state too much.