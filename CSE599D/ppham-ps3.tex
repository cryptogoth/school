\documentclass[12pt]{article}

\usepackage{fullpage}
\usepackage{fancyhdr}
\usepackage{cse-scribe}
\pagestyle{fancy}

\rhead{Problem \thesection\\Page \thepage\\Winter 2006}
\lhead{Paul Pham [ppham@cs.washington.edu]\\CSE 599D: Quantum Computing \\Problem Set 3}

%\renewcommand{\headrulewidth}{0.5pt}
\renewcommand{\footrulewidth}{0pt}

\addtolength{\headheight}{42pt} % make space for the rule
\addtolength{\headsep}{6pt} % make space for the rule

\renewcommand{\labelenumi}{\textbf{\alph{enumi})}}
\renewcommand{\labelenumii}{\textbf{\arabic{enumii}}}
%\renewcommand{\thesection}{\small{Problem \arabic{section}.}}
%  \makeatletter
%   \renewcommand{\section}{\@startsection{section}{1}{0mm}
%   {\baselineskip}%
%   {\baselineskip}{\normalfont\normalsize}}%
%   \makeatother
%\renewcommand{\section}{\@startsection{section}{1}}
\def\qopnamewl@#1{\mathop{\fam\z@#1}\nlimits@}
\def\Exp{\mathop{\rm {E}}}
\def\dist{{\rm dist}\,}

%    Q-circuit version 1.06
%    Copyright (C) 2004  Steve Flammia & Bryan Eastin

%    This program is free software; you can redistribute it and/or modify
%    it under the terms of the GNU General Public License as published by
%    the Free Software Foundation; either version 2 of the License, or
%    (at your option) any later version.
%
%    This program is distributed in the hope that it will be useful,
%    but WITHOUT ANY WARRANTY; without even the implied warranty of
%    MERCHANTABILITY or FITNESS FOR A PARTICULAR PURPOSE.  See the
%    GNU General Public License for more details.
%
%    You should have received a copy of the GNU General Public License
%    along with this program; if not, write to the Free Software
%    Foundation, Inc., 59 Temple Place, Suite 330, Boston, MA  02111-1307  USA

\usepackage[matrix,frame,arrow]{xy}
\usepackage{amsmath}
\newcommand{\bra}[1]{\left\langle{#1}\right\vert}
\newcommand{\ket}[1]{\left\vert{#1}\right\rangle}
    % Defines Dirac notation.
\newcommand{\qw}[1][-1]{\ar @{-} [0,#1]}
    % Defines a wire that connects horizontally.  By default it connects to the object on the left of the current object.
    % WARNING: Wire commands must appear after the gate in any given entry.
\newcommand{\qwx}[1][-1]{\ar @{-} [#1,0]}
    % Defines a wire that connects vertically.  By default it connects to the object above the current object.
    % WARNING: Wire commands must appear after the gate in any given entry.
\newcommand{\cw}[1][-1]{\ar @{=} [0,#1]}
    % Defines a classical wire that connects horizontally.  By default it connects to the object on the left of the current object.
    % WARNING: Wire commands must appear after the gate in any given entry.
\newcommand{\cwx}[1][-1]{\ar @{=} [#1,0]}
    % Defines a classical wire that connects vertically.  By default it connects to the object above the current object.
    % WARNING: Wire commands must appear after the gate in any given entry.
\newcommand{\gate}[1]{*{\xy *+<.6em>{#1};p\save+LU;+RU **\dir{-}\restore\save+RU;+RD **\dir{-}\restore\save+RD;+LD **\dir{-}\restore\POS+LD;+LU **\dir{-}\endxy} \qw}
    % Boxes the argument, making a gate.
\newcommand{\meter}{\gate{\xy *!<0em,1.1em>h\cir<1.1em>{ur_dr},!U-<0em,.4em>;p+<.5em,.9em> **h\dir{-} \POS <-.6em,.4em> *{},<.6em,-.4em> *{} \endxy}}
    % Inserts a measurement meter.
\newcommand{\measure}[1]{*+[F-:<.9em>]{#1} \qw}
    % Inserts a measurement bubble with user defined text.
\newcommand{\measuretab}[1]{*{\xy *+<.6em>{#1};p\save+LU;+RU **\dir{-}\restore\save+RU;+RD **\dir{-}\restore\save+RD;+LD **\dir{-}\restore\save+LD;+LC-<.5em,0em> **\dir{-} \restore\POS+LU;+LC-<.5em,0em> **\dir{-} \endxy} \qw}
    % Inserts a measurement tab with user defined text.
\newcommand{\measureD}[1]{*{\xy*+=+<.5em>{\vphantom{#1}}*\cir{r_l};p\save*!R{#1} \restore\save+UC;+UC-<.5em,0em>*!R{\hphantom{#1}}+L **\dir{-} \restore\save+DC;+DC-<.5em,0em>*!R{\hphantom{#1}}+L **\dir{-} \restore\POS+UC-<.5em,0em>*!R{\hphantom{#1}}+L;+DC-<.5em,0em>*!R{\hphantom{#1}}+L **\dir{-} \endxy} \qw}
    % Inserts a D-shaped measurement gate with user defined text.
\newcommand{\multimeasure}[2]{*+<1em,.9em>{\hphantom{#2}} \qw \POS[0,0].[#1,0];p !C *{#2},p \drop\frm<.9em>{-}}
    % Draws a multiple qubit measurement bubble starting at the current position and spanning #1 additional gates below.
    % #2 gives the label for the gate.
    % You must use an argument of the same width as #2 in \ghost for the wires to connect properly on the lower lines.
\newcommand{\multimeasureD}[2]{*+<1em,.9em>{\hphantom{#2}}\save[0,0].[#1,0];p\save !C *{#2},p+LU+<0em,0em>;+RU+<-.8em,0em> **\dir{-}\restore\save +LD;+LU **\dir{-}\restore\save +LD;+RD-<.8em,0em> **\dir{-} \restore\save +RD+<0em,.8em>;+RU-<0em,.8em> **\dir{-} \restore \POS !UR*!UR{\cir<.9em>{r_d}};!DR*!DR{\cir<.9em>{d_l}}\restore \qw}
    % Draws a multiple qubit D-shaped measurement gate starting at the current position and spanning #1 additional gates below.
    % #2 gives the label for the gate.
    % You must use an argument of the same width as #2 in \ghost for the wires to connect properly on the lower lines.
\newcommand{\control}{*-=-{\bullet}}
    % Inserts an unconnected control.
\newcommand{\controlo}{*!<0em,.04em>-<.07em,.11em>{\xy *=<.45em>[o][F]{}\endxy}}
    % Inserts a unconnected control-on-0.
\newcommand{\ctrl}[1]{\control \qwx[#1] \qw}
    % Inserts a control and connects it to the object #1 wires below.
\newcommand{\ctrlo}[1]{\controlo \qwx[#1] \qw}
    % Inserts a control-on-0 and connects it to the object #1 wires below.
\newcommand{\targ}{*{\xy{<0em,0em>*{} \ar @{ - } +<.4em,0em> \ar @{ - } -<.4em,0em> \ar @{ - } +<0em,.4em> \ar @{ - } -<0em,.4em>},*+<.8em>\frm{o}\endxy} \qw}
    % Inserts a CNOT target.
\newcommand{\qswap}{*=<0em>{\times} \qw}
    % Inserts half a swap gate. 
    % Must be connected to the other swap with \qwx.
\newcommand{\multigate}[2]{*+<1em,.9em>{\hphantom{#2}} \qw \POS[0,0].[#1,0];p !C *{#2},p \save+LU;+RU **\dir{-}\restore\save+RU;+RD **\dir{-}\restore\save+RD;+LD **\dir{-}\restore\save+LD;+LU **\dir{-}\restore}
    % Draws a multiple qubit gate starting at the current position and spanning #1 additional gates below.
    % #2 gives the label for the gate.
    % You must use an argument of the same width as #2 in \ghost for the wires to connect properly on the lower lines.
\newcommand{\ghost}[1]{*+<1em,.9em>{\hphantom{#1}} \qw}
    % Leaves space for \multigate on wires other than the one on which \multigate appears.  Without this command wires will cross your gate.
    % #1 should match the second argument in the corresponding \multigate. 
\newcommand{\push}[1]{*{#1}}
    % Inserts #1, overriding the default that causes entries to have zero size.  This command takes the place of a gate.
    % Like a gate, it must precede any wire commands.
    % \push is useful for forcing columns apart.
    % NOTE: It might be useful to know that a gate is about 1.3 times the height of its contents.  I.e. \gate{M} is 1.3em tall.
    % WARNING: \push must appear before any wire commands and may not appear in an entry with a gate or label.
\newcommand{\gategroup}[6]{\POS"#1,#2"."#3,#2"."#1,#4"."#3,#4"!C*+<#5>\frm{#6}}
    % Constructs a box or bracket enclosing the square block spanning rows #1-#3 and columns=#2-#4.
    % The block is given a margin #5/2, so #5 should be a valid length.
    % #6 can take the following arguments -- or . or _\} or ^\} or \{ or \} or _) or ^) or ( or ) where the first two options yield dashed and
    % dotted boxes respectively, and the last eight options yield bottom, top, left, and right braces of the curly or normal variety.
    % \gategroup can appear at the end of any gate entry, but it's good form to pick one of the corner gates.
    % BUG: \gategroup uses the four corner gates to determine the size of the bounding box.  Other gates may stick out of that box.  See \prop. 
\newcommand{\rstick}[1]{*!L!<-.5em,0em>=<0em>{#1}}
    % Centers the left side of #1 in the cell.  Intended for lining up wire labels.  Note that non-gates have default size zero.
\newcommand{\lstick}[1]{*!R!<.5em,0em>=<0em>{#1}}
    % Centers the right side of #1 in the cell.  Intended for lining up wire labels.  Note that non-gates have default size zero.
\newcommand{\ustick}[1]{*!D!<0em,-.5em>=<0em>{#1}}
    % Centers the bottom of #1 in the cell.  Intended for lining up wire labels.  Note that non-gates have default size zero.
\newcommand{\dstick}[1]{*!U!<0em,.5em>=<0em>{#1}}
    % Centers the top of #1 in the cell.  Intended for lining up wire labels.  Note that non-gates have default size zero.
\newcommand{\Qcircuit}{\xymatrix @*=<0em>}
    % Defines \Qcircuit as an \xymatrix with entries of default size 0em.


\begin{document}
%\newcommand{\ket}[1]{|#1 \rangle}
%\newcommand{\bra}[1]{\langle #1 |}
\newcommand{\braket}[2]{\langle #1|#2 \rangle}
\newcommand{\normtwo}{\frac{1}{\sqrt{2}}}

%%%%%%%%%%%%%%%%%%%%%%%%%%%%%%%%%%%%%%%%%%%%%%%%%%%%%%%%%%%%%%%%%%%%%%%%%%%%%%%
\section{Tsirel'son's Inequality}

\begin{enumerate}

%------------------------------------------------------------------------------
\item % a
%
\begin{eqnarray*}
C^2 & = & (AB + AB' + A'B - A'B')^2 =\\
    & = & A^2B^2 + A'^2BB' + AA'B^2 - AA'BB' +\\
    &   & A^2B'B + A^2B'^2 + AA'B'B - AA'B'^2 +\\
    &   & A'AB^2 + A'ABB' + A'^2B^2 - A'^2BB' +\\
    &   & -A'AB'B - A'AB'^2 - A'^2B'B + A'^2B'^2\\
    & = & I + BB' + AA' - AA'BB' +\\
    &   & B'B + I + AA'B'B - AA' +\\
    &   & A'A + A'ABB' + I - BB' +\\
    &   & -A'AB'B - A'A - B'B + I\\
    & = & 4I - AA'BB' + AA'B'B + A'ABB' - A'AB'B\\
    & = & 4I + AA'(B'B-BB') - (A'A(B'B-BB')\\
    & = & 4I - (AA'-A'A)(BB'-B'B)\\
    & = & 4I - [A,A'][B,B']
\end{eqnarray*}

%------------------------------------------------------------------------------
\item % b
The inequality is saturated below when $MN = 0$.
%
\begin{eqnarray*}
\displaystyle ||M+N||_{\textrm{sup}}
& = & \sup_{\ket{\psi}\ne 0} \frac{||(M+N)\ket{\psi}||}{||\ket{\psi}||}\\
& = & \sup_{\ket{\psi}\ne 0} \frac{\sqrt{\bra{\psi}(M^\dagger+N^\dagger)(M+N)\ket{\psi}}}{||\ket{\psi}||}\\
& \le & \sup_{\ket{\psi}\ne 0} \frac{\sqrt{\bra{\psi}(M^\dagger+N^\dagger+2N^\dagger M^\dagger)(M+N+2MN)\ket{\psi}}}{||\ket{\psi}||}\\
& = & \sup_{\ket{\psi}\ne 0} \frac{\sqrt{\bra{\psi}M^\dagger M\ket{\psi}} + \sqrt{\bra{\psi}N^\dagger N\ket{\psi}}}{||\ket{\psi}||}\\
& = & \sup_{\ket{\psi}\ne 0} \frac{||M\ket{\psi}||+||N\ket{\psi}||}{||\ket{\psi}||}\\
& \le & \sup_{\ket{\psi}\ne 0} \frac{||M\ket{\psi}||}{||\ket{\psi}||} +
    \sup_{\ket{\psi'}\ne 0} \frac{||N\ket{\psi'}||}{||\ket{\psi'}||}\\
& = & ||M||_{\textrm{sup}} + ||N||_{\textrm{sup}}
\end{eqnarray*}
%
Below we use the Cauchy-Schwarz inequality for norms on
inner product spaces.
%
\begin{eqnarray*}
\displaystyle ||MN||_{\textrm{sup}}
& = & \sup_{\ket{\psi}\ne 0} \frac{||(MN)\ket{\psi}||}{||\ket{\psi}||}\\
& = & \sup_{\ket{\psi}\ne 0} \frac{\sqrt{\bra{\psi}(N^\dagger M^\dagger MN)\ket{\psi}}}{||\ket{\psi}||}\\
& \le & \sup_{\ket{\psi}\ne 0} \frac{\sqrt{\bra{\psi}M^\dagger M\ket{\psi}}\cdot\sqrt{\bra{\psi}N^\dagger N\ket{\psi}}}{||\ket{\psi}||}\\
& = & \left(\sup_{\ket{\psi}\ne 0} \frac{||M\ket{\psi}||}{||\ket{\psi}||}\right)\cdot
      \left(\sup_{\ket{\psi'}\ne 0} \frac{||N\ket{\psi'}||}{||\ket{\psi'}||}\right)\\
& = & ||M||_{\textrm{sup}}||N||_{\textrm{sup}}
\end{eqnarray*}

%------------------------------------------------------------------------------
\item % c
Below we use the fact that $A$, $A'$, $B$, and $B'$ are self-adjoint and the
following:
%
\begin{displaymath}
||AA'BB'||_{\textrm{sup}} = \sup_{\ket{\psi}\ne 0} \frac{||AA'BB'\ket{\psi}||}{||\ket{\psi}||} = \frac{\sqrt{\bra{\psi}B'^\dagger B^\dagger A'^\dagger A^\dagger AA'BB'\ket{\psi}}}{||\ket{\psi}||} = 1
\end{displaymath}
%
\begin{eqnarray*}
||C||_{\textrm{sup}} & = & \sqrt{||C^2||_{\textrm{sup}}}\\
& = & \sqrt{||4I - [A,A'][B,B']||_{\textrm{sup}}}\\
& \le & \sqrt{||4I||_{\textrm{sup}} + ||(AA'-A'A)(BB'-B'B)||_{\textrm{sup}}}\\
& \le & \sqrt{||4I||_{\textrm{sup}} + ||(AA'BB')||_{\textrm{sup}} + ||(A'ABB')||_{\textrm{sup}} + ||(AA'B'B)||_{\textrm{sup}} + ||(A'AB'B)||_{\textrm{sup}}}\\
& = & \sqrt{||4I||_{\textrm{sup}} + 4}\\
& = & \sqrt{8} = 2\sqrt{2}
\end{eqnarray*}

\end{enumerate}

\pagebreak

%%%%%%%%%%%%%%%%%%%%%%%%%%%%%%%%%%%%%%%%%%%%%%%%%%%%%%%%%%%%%%%%%%%%%%%%%%%%%%%
\section{A Quantum Error Detecting Code}

\begin{enumerate}

%------------------------------------------------------------------------------
\item % a
Note that $XZ= - ZX$. Then we have the following:
%
\begin{eqnarray*}
S_1S_2 & = & XI \otimes XI \otimes IX \otimes IX\\
       & = & IX \otimes IX \otimes XI \otimes XI = S_2S_1 \\
S_1S_3 & = & XZ \otimes XZ \otimes IZ \otimes IZ\\
       & = & (-ZX) \otimes (-ZX) \otimes ZI \otimes ZI\\
       & = & ZX \otimes ZX \otimes ZI \otimes ZI S_3S_1\\
S_2S_3 & = & IZ \otimes IZ \otimes XZ \otimes XZ\\
       & = & ZI \otimes ZI \otimes (-ZX) \otimes (-ZX)\\
       & = & ZI \otimes ZI \otimes ZX \otimes ZX = S_3S_1\\
\end{eqnarray*}
%------------------------------------------------------------------------------
\item % b
A projector which annihilates all states outside of the subspace defined by
$S_i\ket{\psi} = \ket{psi}$ is as follows:
%
\begin{displaymath}
P = I + (Z \otimes Z \otimes Z \otimes Z) + (XZX\otimes XZX \otimes XZX \otimes XZX) = S_3S_3 + S_3 + S_1S_2S_3S_1S_2
\end{displaymath}
%
%------------------------------------------------------------------------------
\item % c
One basis for the subspace is
%
\begin{eqnarray*}
\ket{\psi_1} & = \normtwo(\ket{0000}+\ket{1111})\\
\ket{\psi_2} & = \normtwo(\ket{0011}+\ket{1100})
\end{eqnarray*}
%
%------------------------------------------------------------------------------
\item % d
Note that $XZX= -Z$ and $ZXZ = -X$. Then we can construct the following
Pauli group operator, which is not a product of $S_i$s and not identity,
which commutes with each $S_i$.
%
\begin{displaymath}
S_p = XZ \otimes Z \otimes Z \otimes XZ
\end{displaymath}
%
\begin{eqnarray*}
S_1S_p & = & XXZ \otimes XZ \otimes Z \otimes XZ\\
       & = & (-XZX)\otimes (-ZX) \otimes Z \otimes XZ\\
       & = & XZX \otimes ZX \otimes Z \otimes XZ = S_pS_1 \\
S_2S_p & = & XZ \otimes Z \otimes XZ \otimes XXZ\\
       & = & XZ \otimes Z \otimes (-ZX) \otimes (-XZX)\\
       & = & XZ \otimes Z \otimes ZX \otimes XZX = S_pS_2\\
S_3S_p & = & ZXZ \otimes ZZ \otimes ZZ \otimes ZXZ\\
       & = & (-XZZ) \otimes ZZ \otimes ZZ \otimes (-XZZ)\\
       & = & XZZ \otimes ZZ \otimes ZZ \otimes XZZ = S_pS_3\\
\end{eqnarray*}
%------------------------------------------------------------------------------
\item % e
If $P=X$, it will anti-commute with any $S_i$ with $Z$ in the same qubit
(e.g. always with $S_3$).
If $P=Y$, it will anti-commute with any $S_i$ with $X$ or $Y$ in the same qubit
(e.g. always with $S_3$ and either $S_1$ or $S_2$).
If $P=Z$, it will anti-commute with any $S_i$ with $X$ in the same qubit
(e.g. either with $S_1$ or $S_2$).
Therefore, an operator with one Pauli matrix $P$ in any qubit will always
anti-commute with at least one $S_i$.
%------------------------------------------------------------------------------
\item % f
$S_i(Q\ket{\psi}) = S_iQ\ket{\psi} = -QS_i\ket{\psi} = -Q\ket{\psi}$
%------------------------------------------------------------------------------
\item % g
If a nontrivial Pauli operator was applied to $\ket{\psi}$ then
$S_i$ will introduce a negative phase and move the state into an
orthogonal subspace (-1 eigenvalue subspace): $S_i\ket{\psi'} = -\ket{\psi'}$.
Otherwise applying $S_i$ will leave the state in the +1 eigenvalue subspace.
\end{enumerate}

\pagebreak

%%%%%%%%%%%%%%%%%%%%%%%%%%%%%%%%%%%%%%%%%%%%%%%%%%%%%%%%%%%%%%%%%%%%%%%%%%%%%%%
\section{Decoherence-Free Subspaces}

\begin{enumerate}
%------------------------------------------------------------------------------
\item %a
The state $\normtwo(\ket{01}-\ket{10})$ is annihilated by the operators
$X_2$, $Y_2$, and $Z_2$.

%------------------------------------------------------------------------------
\item %b
%------------------------------------------------------------------------------
\item %c
%------------------------------------------------------------------------------
\item %d
An example calculation for $X_4$ shows that an amplitude in every term is
cancelled out by an amplitude of the opposite sign in a matching term, due
to the action of the bit flip $X$ gate:

\begin{eqnarray*}
\ket{\psi}_{12}\otimes\ket{\psi}_{34} & =
\normtwo(\ket{01}_{12}-\ket{10}_{12}) \otimes \normtwo(\ket{01}_{34}-\ket{10}_{34})\\
(X\otimes I \otimes I \otimes I)\ket{\psi}_{12}\otimes\ket{\psi}_{34} & =
\normtwo(\ket{11}_{12}-\ket{00}_{12}) \otimes \normtwo(\ket{01}_{34}-\ket{10}_{34})\\
(I\otimes X \otimes I \otimes I)\ket{\psi}_{12}\otimes\ket{\psi}_{34} & =
\normtwo(\ket{00}_{12}-\ket{11}_{12}) \otimes \normtwo(\ket{01}_{34}-\ket{10}_{34})\\
(I\otimes I \otimes X \otimes I)\ket{\psi}_{12}\otimes\ket{\psi}_{34} & =
\normtwo(\ket{01}_{12}-\ket{10}_{12}) \otimes \normtwo(\ket{11}_{34}-\ket{00}_{34})\\
(I\otimes I \otimes I \otimes X)\ket{\psi}_{12}\otimes\ket{\psi}_{34} & =
\normtwo(\ket{01}_{12}-\ket{10}_{12}) \otimes \normtwo(\ket{00}_{34}-\ket{11}_{34})
\end{eqnarray*}

\begin{eqnarray*}
X_4(\ket{\psi}_{12}\otimes\ket{\psi}_{34}) & = & (\normtwo(\ket{11}_{12}-\ket{00}_{12}+\ket{00}_{12}-\ket{11}_{12})\otimes\normtwo(\ket{01}_{34}-\ket{10}_{45})+\\
& & (\normtwo(\ket{01}_{12}-\ket{10}_{12}) \otimes \normtwo(\ket{11}_{34}-\ket{00}_{34}+\ket{00}_{34}-\ket{11}_{34})\\
& = & 0
\end{eqnarray*}

By similar calculations, $X_4$ also annihilates the states
$\ket{\psi}_{13}\otimes\ket{\psi}_{24}$ and
$\ket{\psi}_{14}\otimes\ket{\psi}_{23}$, and by yet other calculations which
are less similar but still pretty close,
$Y_4$ and $Z_4$ also annihilate these three states.

%An example calculation for $Y_4$:

%\begin{eqnarray*}
%(Y\otimes I \otimes I \otimes I)\ket{\psi}_{12}\otimes\ket{\psi}_{34} & =
%\normtwo(-i\ket{11}_{12}-i\ket{00}_{12}) \otimes \normtwo(\ket{01}_{34}-\ket{10}_{34})\\
%(I\otimes Y \otimes I \otimes I)\ket{\psi}_{12}\otimes\ket{\psi}_{34} & =
%\normtwo(i\ket{00}_{12}+i\ket{11}_{12}) \otimes \normtwo(\ket{01}_{34}-\ket{10}_{34})\\
%(I\otimes I \otimes Y \otimes I)\ket{\psi}_{12}\otimes\ket{\psi}_{34} & =
%\normtwo(\ket{01}_{12}-\ket{10}_{12}) \otimes \normtwo(-i\ket{11}_{34}-i\ket{00}_{34})\\
%(I\otimes I \otimes I \otimes Y)\ket{\psi}_{12}\otimes\ket{\psi}_{34} & =
%\normtwo(\ket{01}_{12}-\ket{10}_{12}) \otimes \normtwo(i\ket{00}_{34}+i\ket{11}_{34})
%\end{eqnarray*}

%\begin{eqnarray*}
%Y_4(\ket{\psi}_{12}\otimes\ket{\psi}_{34}) & = & (\normtwo(-i\ket{11}_{12}-i\ket{00}_{12}+i\ket{00}_{12}+i\ket{11}_{12})\otimes\normtwo(\ket{01}_{34}-\ket{10}_{45})+\\
%& & (\normtwo(\ket{01}_{12}-\ket{10}_{12}) \otimes \normtwo(-i\ket{11}_{34}-i\ket{00}_{34}+i\ket{00}_{34}+i\ket{11}_{34})\\
%& = & 0
%\end{eqnarray*}

%By similar calculations, $Y_4$ also annihilates the states
%$\ket{\psi}_{13}\otimes\ket{\psi}_{24}$ and
%$\ket{\psi}_{14}\otimes\ket{\psi}_{23}$.

%An example calculation for $Z_4$:

%\begin{eqnarray*}
%(Z\otimes I \otimes I \otimes I)\ket{\psi}_{12}\otimes\ket{\psi}_{34} & =
%\normtwo(-\ket{11}_{12}-\ket{00}_{12}) \otimes \normtwo(\ket{01}_{34}-\ket{10}_{34})\\
%(I\otimes Z \otimes I \otimes I)\ket{\psi}_{12}\otimes\ket{\psi}_{34} & =
%\normtwo(\ket{00}_{12}+\ket{11}_{12}) \otimes \normtwo(\ket{01}_{34}-\ket{10}_{34})\\
%(I\otimes I \otimes Z \otimes I)\ket{\psi}_{12}\otimes\ket{\psi}_{34} & =
%\normtwo(\ket{01}_{12}-\ket{10}_{12}) \otimes \normtwo(-\ket{11}_{34}-\ket{00}_{34})\\
%(I\otimes I \otimes I \otimes Z)\ket{\psi}_{12}\otimes\ket{\psi}_{34} & =
%\normtwo(\ket{01}_{12}-\ket{10}_{12}) \otimes \normtwo(\ket{00}_{34}+\ket{11}_{34})
%\end{eqnarray*}

%\begin{eqnarray*}
%Z_4(\ket{\psi}_{12}\otimes\ket{\psi}_{34}) & = & (\normtwo(-\ket{11}_{12}-\ket{00}_{12}+\ket{00}_{12}+\ket{11}_{12})\otimes\normtwo(\ket{01}_{34}-\ket{10}_{45})+\\
%& & (\normtwo(\ket{01}_{12}-\ket{10}_{12}) \otimes \normtwo(-\ket{11}_{34}-\ket{00}_{34}+\ket{00}_{34}+\ket{11}_{34})\\
%& = & 0
%\end{eqnarray*}

%By similar calculations, $Z_4$ also annihilates the states
%$\ket{\psi}_{13}\otimes\ket{\psi}_{24}$ and
%$\ket{\psi}_{14}\otimes\ket{\psi}_{23}$.

%------------------------------------------------------------------------------
\item %e
We can expand each of the three states into the following forms:

\begin{eqnarray*}
\ket{\psi}_{12}\otimes\ket{\psi}_{34} & = & \normtwo(\ket{01}_{12}-\ket{10}_{12}) \otimes \normtwo(\ket{01}_{34}-\ket{10}_{34})\\
 & = & \frac{1}{2}(\ket{0101} - \ket{1001} - \ket{0110} + \ket{1010})\\
\ket{\psi}_{13}\otimes\ket{\psi}_{24} & = & \normtwo(\ket{01}_{13}-\ket{10}_{13}) \otimes \normtwo(\ket{01}_{24}-\ket{10}_{24})\\
 & = & \frac{1}{2}(\ket{0011} - \ket{1001} - \ket{0110} + \ket{1100})\\
\ket{\psi}_{14}\otimes\ket{\psi}_{23} & = & \normtwo(\ket{01}_{14}-\ket{10}_{14}) \otimes \normtwo(\ket{01}_{23}-\ket{10}_{23})\\
 & = & \frac{1}{2}(\ket{0011} - \ket{1010} - \ket{0101} + \ket{1100})
\end{eqnarray*}

We can then write the following linear combination, with
$\alpha_1 = 1$, $\alpha_2 = -1$, $\alpha_3 = 1$,  which sums to 0.

\begin{displaymath}
\alpha_1(\ket{\psi}_{12}\otimes\ket{\psi}_{34}) +
\alpha_2(\ket{\psi}_{13}\otimes\ket{\psi}_{24}) +
\alpha_3(\ket{\psi}_{14}\otimes\ket{\psi}_{23}) = 0
\end{displaymath}

Therefore these states are not linearly independent.

%------------------------------------------------------------------------------
\item %f
The following two states are a basis for the subspace, using Gram-Schmidt
on the three states in the previous part.

\begin{displaymath}
\{\ket{\phi_1}, \ket{\phi_2}\} = \{\ket{\psi}_{12}\otimes\ket{\psi}_{34},
\ket{\psi}_{14}\otimes\ket{\psi}_{23}\}
\end{displaymath}

%------------------------------------------------------------------------------
\item %g

\end{enumerate}

\end{document}
