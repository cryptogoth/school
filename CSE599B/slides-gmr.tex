\documentclass[10pt]{beamer}
\mode<beamer>{%
\usetheme[hideothersubsections,right,width=22mm]{Goettingen}
}
\title{How to Play Any Mental Game}
\subtitle{A Completeness Theorem for Protocols With Honest Majority}
\author[[GMR], P. Pham]{[Goldreich, Micali, Wigderson 1987]\\Paul Pham}
\institute{UW CSE}
%\titlegraphic{\includegraphics[width=20mm]{USTL}}
\date{March 2006}

\usepackage{eepic}
\usepackage{color}
\usepackage{rotating}
\usepackage{latexsym} % For \Diamond

\begin{document}
\begin{frame}<handout:0>
\titlepage
\end{frame}

%==============================================================================
\section{Introduction}

%------------------------------------------------------------------------------
\begin{frame}
\frametitle{Overview}

\textbf{Keywords:}
Secure multi-party computation

\textbf{Main Result:}
A meta-algorithm that can implement any multi-player game
\textbf{correctly} and \textbf{privately} assuming only trapdoor functions.

\begin{itemize}
\item A game example: poker
\item Definitions: TM-game, game network, TM-game solver, passive/malicious adversaries.
\item How it works: bird's eye view of protocol
\end{itemize}

\end{frame}

%==============================================================================
\section{Poker Example}
%------------------------------------------------------------------------------
\begin{frame}
\frametitle{Poker Example}

\begin{itemize}
\item Players are dealt a \textit{hand} of private cards.
\item In each betting round, players must match/raise the current bet or fold.
\begin{enumerate}
\item Hands can develop in between betting rounds.
\end{enumerate}
\item At end of last betting round, player with best combination wins.
\end{itemize}

Poker is a game with \textit{incomplete} information because players only
have partial knowledge of deck.
It is \textit{playable} (self-enforceable) in a purely physical model.

\begin{itemize}
\item Cards are opaque and have the same back.
\item Tables are not too reflective.
\item Finite and unforgeable chips.
\end{itemize}

Are all games playable? If so, in what model?

\end{frame}

%------------------------------------------------------------------------------
\frametitle{Secure Multiparty Games}

\begin{frame}

\subsection{More Specifically}

\begin{description}
\item[Playable]: an $n$ player game that can be carried out with only
$n$ players (no separate poker dealer).
\item[Honest]: following protocol (rules of the game).
\end{description}

\subsection{[GMR] says yes.}
In fact, all games are playable in a \textit{computational complexity} model,
as long as at least half the players are honest.

\subsection{Encrypted Poker?}
We are interested in special kinds of games which meet two constraints:

\begin{description}
\item[Correctness] following the protocol/rules.
\item[Privacy] no player can learn anything more than output and individual inputs.
\end{description}

These can be called \textit{secure multi-party games}.

%\begin{description}
%\item[Game]:
%\begin{itemize}
%\item set of states $S$ (poker hands, money, pot, bet values, betting rounds)
%\item a current state $\sigma \in S$ (snapshot of poker game)
%\item set of moves to transform current state $M$ (rules for changing hands and placing bets)
%\item partial information function $K_i(\sigma)$ (individual hands)
%\item payoff function $p(\sigma_\omega)$ (which hand is better)
%\end{itemize}

\end{frame}

%==============================================================================
\section{Definitions}

%------------------------------------------------------------------------------
\begin{frame}
\frametitle{TM-games and game networks}

A \textit{TM-game} is a pair of $(\langle M\rangle ,1^k)$ where $M$
encodes the desired game and $k$ is the security parameter.
\begin{itemize}
\item Special case of general game with empty initial state and where all
player moves once, simultaneously.
\item $M$ takes inputs all of equal length $l$.
\item Running time of $M$ is $T(l) < k$
\end{itemize}

A \textit{game network} is the theoretical ``hardware model''
 (an $n$-player TM) for running \textit{solvers} for TM-games.

\begin{itemize}
\item $n$ Turing machines, labeled with a name $i$.
\item $n(n-1)$ pairwise, private, ``directed'' tapes (write-only $\rightarrow$ read-only)
\item A common read-only input tape.
\item A common clock.
\end{itemize}

\end{frame}

%------------------------------------------------------------------------------
\begin{frame}
\frametitle{Adversaries}

\begin{description}
\item[Passive adversary]:
\begin{itemize}
\item must follow protocol (correctness)
\item but can also run extra computation ``on-the-side'' to try to break
privacy
\item allowed, because follow correctness and we can enforce privacy.
\end{itemize}

\item[Malicious adversary]:
\begin{itemize}
\item can deviate from protocol at any time to break privacy.
\item e.g. can return random or manufactured values other than its input.
\item allowed only in minority, because we can enforce both correctness and privacy.
\end{itemize}

\end{description}

\end{frame}

%------------------------------------------------------------------------------
\begin{frame}
\frametitle{TM-Game Solvers}

\begin{frame}

\textbf{For passive adversaries only}

A \textit{TM-game solver} is a distributed polytime algorithm which runs
on a game network of size $n$ and on common inputs
$(\langle M \rangle, 1^k)$ and private inputs $(x_1,\ldots,x_n)$:


\end{frame}

%------------------------------------------------------------------------------
\begin{frame}
\frametitle{Concepts Needed From Course}

\begin{description}

\item[Computational indistinguishability]
If the probability of any PPT judge to tell two random variables apart is
negligible in $k$.

\item[
\item[One-way function family]
A family $\mathcal{F} = \{f_i\}$ such that each function $f_i$ on
parameter $k$ is
\begin{itemize}
\item easy to index
\item easy to sample from its domain
\item hard-to-invert (prob. of any PPT adversary is negl.)
\end{itemize}

\item[Trapdoor function family]
Same as previous but additionally each $f_i$ is easy-to-invert with
the secret trapdoor $f_i^{-1}$.

\item[Trapdoor permutation]
Same as one-way permutation but with requirement that trapdoor
(inverse permutation) exists.

\item[Hardcore bits]
A bit $B(x)$ is hardcore with respect to a one-way function $f$ if
given only $f(x)$, it is comp. indist. from a random bit.

\end{description}

\end{frame}

%------------------------------------------------------------------------------
\begin{frame}
\frametitle{How Does It Work?}

\begin{enumerate}
\item Run \textit{engagement protocol} to force all players to \textit{commit}
to their inputs.
\item Run \textit{TM-game solver for passive adversaries} to execute the
TM-game $M$.
\end{enumerate}

\end{frame}

%==============================================================================
\section{TM Game Solver}

%------------------------------------------------------------------------------
\begin{frame}
\frametitle{TM-Game Solver}

Uses result in [Barrington], all straight-line programs can be encoded as
permutations $\sigma$ in $S_5$.

A \textit{straight-line} program consists only of evaluating
Boolean functions of two inputs, with no conditional branches or loops.

\begin{enumerate}
\item Convert $M$ into an equivalent circuit $C$.
\item Convert $C$ into a straight-line program $P$.
\item Encode $P$ as follows:
\begin{itemize}
\item Denote 0,1 as two special $\sigma_0$ and $\sigma_1$
\item Function outputs are in $S_5$
\item For each of three kinds of instructions, we perform:
\begin{itemize}
\item multiplying a $\sigma$ by a constant $c$.
\item multiplying the inverse $\sigma$ by a constant $c$.
\item multiplying a $\sigma$ by variable $r$.
\end{itemize}
\item For each player $i$:
\begin{enumerate}
\item For each private bit $b$ of $x_i$:
\begin{enumerate}
\item Encode $b$ as $\sigma \in S_5$.
\item Choose $n-1$ random $\sigma_i$'s.
\item Compute $\sigma_n = (\sigma_1\cdots \sigma_{n-1})^{-1}\sigma$.
\item Give $(j,\sigma_j)$ to player $j$.
\end{enumerate}
\end{enumerate}

\end{frame}

%==============================================================================
\section{Honest Majorities}

%------------------------------------------------------------------------------
\begin{frame}
\frametitle{Engagement Protocol}

This is run at the beginning to force all malicious adversaries to be
passive adversaries.

\begin{enumerate}
\item For each player $i$, a protocol is performed such that no minority
can predict any bit of $x_i$ better than randomly.
\item 
\end{enumerate}


\end{frame}

\end{document}
