\documentclass[12pt]{article}

\usepackage{fullpage}
\usepackage{fancyhdr}
\usepackage{cse-scribe}
\usepackage{xypic}
\pagestyle{fancy}

\rhead{Problem \thesection\\Page \thepage\\Spring 2006}
\lhead{Paul Pham [ppham@cs.washington.edu]\\CSE 544: Databases\\Homework 4}

%\renewcommand{\headrulewidth}{0.5pt}
\renewcommand{\footrulewidth}{0pt}

\addtolength{\headheight}{42pt} % make space for the rule
\addtolength{\headsep}{6pt} % make space for the rule

\renewcommand{\labelenumi}{(\alph{enumi})}
\renewcommand{\labelenumii}{\roman{enumii}.}
%\renewcommand{\thesection}{\small{Problem \arabic{section}.}}
%  \makeatletter
%   \renewcommand{\section}{\@startsection{section}{1}{0mm}
%   {\baselineskip}%
%   {\baselineskip}{\normalfont\normalsize}}%
%   \makeatother
%\renewcommand{\section}{\@startsection{section}{1}}
\def\qopnamewl@#1{\mathop{\fam\z@#1}\nlimits@}
\def\Exp#1{\mathop{\rm {E}}[#1]}
\def\argmax{{\rm arg} \mathop{\rm {max}}}
\def\argmin{{\rm arg} \mathop{\rm {min}}}
\def\dist{{\rm dist}\,}
\def\PR#1{\mathop{\rm {Pr}}\left[#1\right]}

\newenvironment{lscommand}%
    {\nopagebreak\par\small\addvspace{3.2ex plus 0.8ex minus 0.2ex}%
     \vskip -\parskip
     \noindent%
     \begin{tabular}{|l|}\hline\rule{0pt}{1em}\ignorespaces}%
    {\\\hline\end{tabular}\par\nopagebreak\addvspace{3.2ex plus 0.8ex
        minus 0.2ex}%
     \vskip -\parskip}

\begin{document}

%%%%%%%%%%%%%%%%%%%%%%%%%%%%%%%%%%%%%%%%%%%%%%%%%%%%%%%%%%%%%%%%%%%%%%%%%%%%%%%
% Problem 1
\section{Timestamp concurrency manager}

\begin{enumerate}
%------------------------------------------------------------------------------
\item % Part A
$w_1(C)$ is \textbf{accepted}
because Transaction 2 should be rolled back and restarted.

%------------------------------------------------------------------------------
\item % Part B
Transaction 1 is \textbf{rolled back}
and restarted because it wrote too late to B.

%------------------------------------------------------------------------------
\item % Part C
$w_1(A)$ is \textbf{ignored}
because Transaction 3 has also written to $A$ and has
already committed.

%------------------------------------------------------------------------------
\item % Part D
Transaction 1 is \textbf{delayed}. If Transaction 2 commits,
$w_1(A)$ can be ignored, but not otherwise in general.

\end{enumerate}

%%%%%%%%%%%%%%%%%%%%%%%%%%%%%%%%%%%%%%%%%%%%%%%%%%%%%%%%%%%%%%%%%%%%%%%%%%%%%%%
% Problem 2
\section{Logical plans}

\begin{enumerate}
%------------------------------------------------------------------------------
\item % Part A

\begin{displaymath}
\xymatrix{
 & & \Pi_{a,sum(T.d)} \ar[d] & \\
 & & \sigma_{sum(S.h) > 10} \ar[d] & \\
 & & \gamma_{a, sum(T.d), sum(S.h)} \ar[d] & \\
 & & \Join_{S.c = T.c} \ar[dl] \ar[ddr] &\\
 & \Join_{R.b=S.b} \ar[dl] \ar[ddr] & &\\
\sigma_{f=5} \ar[d] & &          & \sigma_{g=9} \ar[d] \\
R(a,b,f)            & & S(b,c,h) & T(c,d,g)
}
\end{displaymath}

%------------------------------------------------------------------------------
\item % Part B
\begin{center}
\begin{tabular}[t]{ll}
\textsc{Select} & $R.a, T.g$\\
\textsc{From}   & $R,S,T$\\
\textsc{Where}  & $(R.a > 5)$ and $(T.g < 9)$ and $(R.b=S.b)$ and $(S.c=T.c)$
\end{tabular}
\end{center}

\end{enumerate}

\pagebreak

%%%%%%%%%%%%%%%%%%%%%%%%%%%%%%%%%%%%%%%%%%%%%%%%%%%%%%%%%%%%%%%%%%%%%%%%%%%%%%%
% Problem 3
\section{Drinkers, beers, bars}

\begin{enumerate}
%------------------------------------------------------------------------------
\item % Part A

\begin{tabular}[t]{ll}
\textsc{Select Distinct} & $drinkers$\\
\textsc{From}   & Likes $\ell$, Frequents $f$, Serves $s$\\
\textsc{Where}  & $(\ell.drinker = f.drinker)$ and \\
                & $(f.bar = s.bar)$ and \\
                & $(\ell.beer = s.beer)$ and \\
                & \textsc{Not Exists} (
\begin{tabular}[t]{ll}
\textsc{Select} & *\\
\textsc{From}   & Frequents $f_2$, Serves $s_2$\\
\textsc{Where}  & $(l.drinker = f_2.drinker)$ and \\
                & $(f_2.bar = s_2.bar)$ and \\
                & $(l.beer \ne s_2.beer)$ )
\end{tabular}
\end{tabular}

%------------------------------------------------------------------------------
\item % Part B

\begin{tabular}[t]{ll}
\textsc{Select Distinct} & $drinkers$\\
\textsc{From}   & Likes $\ell$, Frequents $f$, Serves $s$\\
\textsc{Where}  & $(\ell.drinker = f.drinker)$ and \\
                & $(f.bar = s.bar)$ and \\
                & $(\ell.beer = s.beer)$ and \\
                & \textsc{Not Exists} (
\begin{tabular}[t]{ll}
\textsc{Select} & *\\
\textsc{From}   & Frequents $f_2$, Serves $s_2$\\
\textsc{Where}  & $(l.drinker = f_2.drinker)$ and \\
                & $(f_2.bar = s_2.bar)$ and \\
                & $(l.beer = s_2.beer)$ )
\end{tabular}
\end{tabular}

\end{enumerate}

\pagebreak

%%%%%%%%%%%%%%%%%%%%%%%%%%%%%%%%%%%%%%%%%%%%%%%%%%%%%%%%%%%%%%%%%%%%%%%%%%%%%%%
% Problem 4
\section{Selectivity estimation}

\begin{enumerate}
%------------------------------------------------------------------------------
\item % Part A
We abbreviate the event that field $A=a$ as simply $A$ and similarly for
the events $B=b$ and $C=c$.
We want to determine the following probability.
%
\begin{eqnarray*}
\PR{ABC} & = & \PR{C}\PR{B\mid C}\PR{A\mid BC}\\
         & = & \PR{C}\PR{B\mid C}\PR{A\mid B}
\end{eqnarray*}
%
The first step follows from the chain rule for conditional probabilities
and the second step follows from the conditional independence of $A$ and $C$
given $B$. We then have the following:
%
\begin{eqnarray*}
& \PR{B \mid C} = \PR{BC} / \PR{C} = p_{23} / p_{3} \\
& \PR{A \mid B} = \PR{AB} / \PR{B} = p_{12} / p_{2} \\
& \PR{ABC} = p_{23} p_{12} / p_{2} \\
\end{eqnarray*}
%
The answer is the final expression multiplied by the size of the
relation ($n$) to get the estimated result size.

%------------------------------------------------------------------------------
\item % Part B

If the second \textsc{AND} is replaced with an \textsc{OR}, we can apply
the sum rule for probabilities and use the expression in the previous
part to subtract the intersection.
%
\begin{displaymath}
\PR{AB \cup C} = \PR{C} + \PR{B}\PR{A\mid B} - \PR{ABC}
\end{displaymath}
%
This expression is simply $p_{3} + p_{12} - (p_{23} p_{12} / p_{2})$,
multiplied by the size of the
relation ($n$) to get the estimated result size.

\end{enumerate}

\pagebreak

%%%%%%%%%%%%%%%%%%%%%%%%%%%%%%%%%%%%%%%%%%%%%%%%%%%%%%%%%%%%%%%%%%%%%%%%%%%%%%%
% Problem 5
\section{Plan search space}

\begin{enumerate}
%------------------------------------------------------------------------------
\item % Part A
The number of left-linear chain join plans among $n$ relations where order
matters is equal to the sum of ways to join when $R_1$ is chosen
first, when $R_2$ is chosen first, etc., up until $R_n$ is chosen first.
If $R_i$ is chosen first, then the remaining $n-1$ choices are fixed
(ways of ``moving right'') except for $i-1$ choices (ways of ``moving left'')
which must always appear in decreasing order ($R_{i-1}, R_{i-2}, \ldots$).

\begin{displaymath}
\sum_{i=1}^{n} \binom{n-1}{i-1} = \sum_{i=0}^{n-1} \binom{n-1}{i} = 2^{n-1}
\end{displaymath}

The last step follows from the Binomial Theorem by setting $x=y=1$.

\begin{displaymath}
\sum_{j=0}^{n} \binom{n}{j}x^{n-j}y^j = (x+y)^n
\end{displaymath}

%------------------------------------------------------------------------------
\item % Part B
The number of star join plans among $n$ relations is the number of ways of
permuting the relations: $n!$.

%------------------------------------------------------------------------------
\item % Part C
The number of left-linear chain subplans considered by dynamic programming is
equal to the sum of all ``joins'' of 1 relation in
the base case ($n$) plus the number of ways to inductively compose a join
of $k$ relations from a chain join of $k-1$ relations in the fixed 
order (two ways).
%
\begin{eqnarray*}
n + \sum_{i=1}^n \sum_{k=1}^{n-i} 2 & = & n + \sum_{i=1}^n 2(n-i)\\
                                    & = & n + 2\sum_{i=0}^{n-1} i \\
                                    & = & n + 2(\frac{(n-1)n}{2})\\
                                    & = & n^2
\end{eqnarray*}

%------------------------------------------------------------------------------
\item % Part D
Dynamic programming does not reduce the number of star chain subplans
considered which is still $n!$.

\end{enumerate}

\end{document}
