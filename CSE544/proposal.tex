\documentclass{toc}

\usepackage{times}
\usepackage{fullpage}
\usepackage{eprint}

%\tocdetails{%
%   volume=0, number=5, year=2005, firstpage=78,
%   received={October 15, 2004}, 
%   published={December 31, 2004}
%}

\theoremstyle{plain}
\newtheorem{idea}[theorem]{Idea}
%\newtheorem{idea}{Idea}[section]

\theoremstyle{definition}
\newtheorem{thought}[theorem]{Thought}
%\newtheorem{thought}{Thought}[section]

\DeclareMathOperator{\cov}{\textsf{Cov}}

\title{CSE 544 Project Proposal\\Spring 2006}
\author{Paul T. Pham\\ppham@cs.washington.edu}

\runningauthor{\begin{minipage}{.9\textwidth}\centering P. Pham,
    \end{minipage}}
\runningtitle{Typesetting guidelines}
%\copyrightauthor{Alex Rusell and the extremely unnecessaily 
%large ficticious family of authors}

\begin{document}

\begin{center}
\LARGE{CSE 544 Project Proposal, Spring 2006}\\
\vspace{\baselineskip}
\Large{Paul T. Pham\\ (ppham@cs.washington.edu)}
\end{center}

\section{Abstract}

The new field of quantum computing has given us many speedups over classical
solutions, especially for searching functions, finding function collisions,
counting solutions, and related problems. These problems are generally
associated with the term ``quantum database,'' although there is no
direct connection to the relational model of classical databases.
In this project, I propose to summarize the current state-of-the-art in
quantum search, determine which problems are amenable to a quantum
solution, formalize a model and a matching architecture for these
problems, and simulate/measure its operation in software.
Under this model, I will consider the efficiency of basic operations such
as lookup, join, and difference; the feasibility and
effects of indexing; and the possible application to harder problems such as
nearest-neighbor search, subset search, and probabilistic query evaluation.

\section{Introduction}

In recent years, quantum computing has emerged as a more general and
powerful model of computation than the more familiar digital logic.
Using the properties of quantum physics, we can solve many problems which are
provably hard on classical computers, such as
perfectly secure cryptographic communication \cite{lc99},
symmetric private information retrieval without shared randomness \cite{kw04},
and the Byzantine generals problem for fault-tolerant distributed computation
\cite{cabello03}.

One of the oldest results in this field, and one with the most applications,
is quantum
search in an unordered list due to Grover \cite{grover96}. Although the
literature often refers to this and related problems as
``quantum database search,'' it is more accurate to say they deal with
quantum function search. The model consists of a list of $N$ items taken
from a set $B$ which are stored at addresses from a set $A$. The items
are accessed through a ``database function'' $f: A \rightarrow B$ which
takes addresses to items. A item $b \in B$ is said to exist in the database
if for some $a \in A$, $f(a) = b$.
Given a target key $k \in B$, search complexity is
defined as the number of oracle queries to $f$ needed to determine the
address of $k$ in the database, if it exists. We will denote the number of
quantum queries needed as $Q(N)$ and the number of deterministic, classical
queries as $D(N)$, for some bounded error probability (say $< \frac{1}{2}$).
This can be amplified in either case by $O(\log{1/\epsilon})$ independent
trials to bring the error down to $\epsilon$ at the cost of greater query
complexity. Although there are obvious limitations to this model,
it still allows
many surprisingly good quantum speedups; both are discussed in Section 2.
In Section 3 we outline the proposed tasks for this project, with a timeline
and list of deliverables in Section 4. Finally, we conclude with some of the
deeper motivation and implications of this project in Section 5.

\section{Related Work}

In Grover's original model, the list is unsorted, $N$ is
a power of two ($2^n$), the addresses are binary numbers
($A = \{0,1\}^n$), the items are Boolean values ($B = \{0,1\}$), and
$f$ is 1 on a single target state and 0 everywhere else.
In the classical case,
$D(N) \in O(N)$ while Grover showed that $Q(N) \in O(\sqrt{N})$
(in fact it is optimal \cite{grover98}).
There are several natural variations. For a sorted list,
$D(N) \in O(\log{n})$ using a binary (or bounded-fanout) search, while
only a constant speedup is possible: $Q(N) \ge \log{n}/12 - O(1)$
\cite{ambainis99}.
For multiple ($k \ge 1$) target items, generalized from 1 bit to $m$ bits each,
to find any one solution in an unsorted database takes $D(N) \in O(N/k)$ and
$Q(N) \in O(\sqrt{N/k})$
queries, even when the number $k$ is not known. To
find all $k$ items takes $Q(N) \in O(\sqrt{Nk})$ versus $D(N) \in O(N)$
\cite{bbht98}.
For a partial search that only determines the first $\ell$ bits of an address,
$Q(\ell, N) \le \frac{\pi}{4}(1 - c/\sqrt{2^\ell})\sqrt{N}$ for some
constant $c$, a sliding
optimization that is not available to the classical case \cite{gr04}.
To find a single collision (join) between two unordered databases of size
$N$,
$Q(N) \in O(N^{3/4}\log{N})$ versus $D(N) \in O(N\log{N})$, while in the
ordered case $Q(N) \in O(N^{1/2}\log{N})$
\cite{heiligman00}.

While most of these results are quite promising, there are some
important limitations. First of all, the true complexity of a problem must
take into account the overhead of the query itself. So far we have ignored
this, but the quantum and classical cases are not equal with respect to
evaluating $f$. A row, or tuple, in a classical relation can be accessed
in $O(1)$ time with simple pointer lookups. In the quantum case, the
database function must be encoded into a unitary transformation: for
$n$ quantum bits, this generally requires $\Omega(n)$ quantum operations
as preprocessing. Furthermore, classical databases can store arbitrary
relations on $n$-bit addresses taking up $O(2^n)$ space. The
``database functions'' $f$ generally considered in the literature are very
restricted and highly structured; for example, they have a compact
representation, such as a $k$-SAT formula, even though their solution space
is the same $O(2^n)$ size as the classical case.
Moreover, quantum algorithms often require greater
overhead in terms of preprocessing, postprocessing, and error handling.

Above, we have seen the ambiguity associated with the terms ``database''
and ``function''.
At this point, it is hard to compare quantum and classical databases without
coming up with a formal quantum model and exploring the implications of
that model in terms of efficient algorithms and relationship to other
database problems. This comprises the project proposal, which we present in
the next section.

\section{Proposal}

Given the previous discussion, it is clear that quantum computers and
classical computers each have their own advantages and disadvantages.
Furthermore, it is unlikely that quantum computers will ever be suitable
for user interface tasks, deterministic control, networking, office work,
or everyday computing tasks. Therefore, any feasible database model hoping
to exploit quantum search advantages must be a hybrid model.

The prospects for a quantum speedup for lookup and indexing are grim,
but there is still hope that a quantum database can be justified by
faster algorithms for more complex operations.
As the first task of this project, we will conduct further research to
determine how quantum search can
be applied to classical database problems: for example, subset search,
nearest-neighbor search, and probabilistic query evaluation.

As the second task of this project, we will devise a model given what is
known about the current state-of-the-art in search algorithms and
useful applications from database research. There may be several such models,
consisting primarily of storage and control components, as well as the data
and query themselves, any of which may
be classical or quantum. We will also conjecture possible applications
for quantum data and quantum queries; if these are compelling, they alone
would justify a quantum database.

As a third task, we will design a concrete implementation of the abstract
model using what is currently known today about quantum and classical
hardware. As the final task, we will simulate the quantum components of
our model using commonly available simulation software and analyze the
results against the theoretical bounds.

\section{Results and Timeline}

\begin{center}
\begin{tabular}{|l|l|}
\hline
\textbf{Date} & \textbf{Milestone}\\
\hline
April 17 & Proposal finalized.\\
May   1  & Background research complete.\\
May   8  & Formal models complete.\\
May   15 & Architecture design complete.\\
May   22 & Simulation and analysis complete.\\
May   29 & Project report and presentation complete.\\
May   31 & Project presentation.\\
\hline
\end{tabular}
\end{center}

\section{Conclusion}

It is difficult for a programming project to be more ambitious than our
above outline given the experimental status of quantum computing. The many
points left vague in this proposal reflect both the author's inexperience and
the lack of previous work on this topic. There is a chance of failure, that
no models will be feasible or no quantum search algorithm will be efficient
on a realistic architecture. However, we believe this risk is outweighed by the
promise of this research and that even modest results are significant in a
new area.

Thus, we hope to serve two main purposes beyond fulfilling the requirements
of a course project. The first is to increase awareness and collaboration
between the relational database community and the quantum algorithms
community. There are many applications for real-life databases
which would help motivate the young quantum computing field if they were
cast in the right model. Likewise classical problems would
benefit from efficient quantum solutions if they were translated into a
realistic, implementable architecture.

The second goal is personal, and that is to help focus the author's
long-term research interests on a specific area with useful applications
and good potential for a quals project.
Using this work as a foundation, we hope to produce
publishable work at the intersection of traditional computer science theory
and architecture and its newer quantum cousin. Having spent my previous
career on the experimental side, I am convinced that quantum computing will
be a reality within my lifetime. It will not only require pushing from
engineers and experimentalists, but pulling from theorists and dreamers
(and I hope to be both).

\bibliography{proposal}
\bibliographystyle{tocplain}

\end{document}
