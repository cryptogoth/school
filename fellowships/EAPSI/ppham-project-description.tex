\documentclass{article}

\usepackage{eprint}

\setlength{\textheight}{8.5in}	% more margin hacking.
%\addtolength{\textwidth}{1in}	% same here....
\addtolength{\topmargin}{-0.75in}	% repairing LaTeX's huge margins...
%\addtolength{\hoffset}{-0.75in}	% repairing LaTeX's huge margins...

\begin{document}

\begin{center}
\LARGE{2010 NSF EAPSI Project Description}\\
\Large{Paul Pham}
\end{center}


\section{Host Researcher and Institution}

The EAPSI program would give me a valuable opportunity to understand the
international
nature of scientific collaboration by allowing me to form new, personal
connections to the research community in the Pacific Rim.
If awarded this
fellowship, I would travel to Australia for the first time, which is one of
the world's centers for my chosen field of quantum computing. Australia is home
to the ARC Centre of Excellence for Engineered Quantum Systems (EQuS),
a multi-institution, international
collaboration for engineering quantum systems and technology.
In addition, one of the EQuS nodes is especially well-suited to hosting my
fellowship, the University of Sydney.
The Quantum Science Research Group there contains strong researchers 
including Drs. Stephen Bartlett and Andrew Doherty on the
side of theory as well as Drs. Michael Biercuk and David Reilly on the side of
experiment. I would be able to contribute my experience in electronics
engineering and at the same time learn more about the underlying physics
behind quantum computing from these experts in the field.

Dr. Michael J. Biercuk would be an ideal host due to his past experience at NIST
Boulder in achieving high-fidelity quantum control over trapped ions
\cite{BUVSIB2009a}.
In particular, he has developed advanced experimental techniques for 
suppressing decoherence-induced error rates by several orders of magnitude using
any qubit technology, not just trapped ions.
Dr. Biercuk is further pursuing this research
at the University of Sydney where he is Senior Lecturer and head of the
newly-established Quantum Control Laboratory.
This is where I propose to carry out my fellowship in
dealing with the central problem of suppressing quantum noise.

\section{Quantum Quantum Noise}

%Quantum computing is a relatively new field which seeks to harness the
%quantum mechanical properties of nature at a tiny scale to do useful
%computation at a human scale.
%First proposed by Richard Feynman in 1982 as a way of simulating
%quantum systems \cite{Feynman1982}, quantum computing has
%received a great amount of recent interest due to both theoretical discoveries
%and experimental verification.
%It turns out that by using these newly
%discovered computational resources in nature, such as creating a superposition of
%states, entanglement, and efficient randomness, we can solve many important
%problems faster than known non-quantum (classical) solutions.
%Famous
%examples include Shor's algorithm for factoring integers and finding
%discrete logarithms, which breaks RSA and many
%other widely-used cryptosystems \cite{Shor1995};
%Grover's unstructured search \cite{Grover1996}; 
%a random walk on a graph \cite{Childs2003}; solving systems of
%linear equations \cite{Harrow2009}; and simulating molecular energy levels
%\cite{Aspuru-Guzik2005}. Some of these have even been implemented
%in exciting experimental results \cite{Vandersypen2001} \cite{Schmitz2009}
%\cite{Friedenauer2008},
%giving us more confidence that some day
%we will be able to run these algorithms on a reliable, scalable quantum
%computer.

In order to realize the benefits of any quantum technology, especially
quantum information processing and quantum computing, we need to deal with the
practical problems of fault-tolerance and noise.
Any quantum system must be shielded from
unintended couplings to its environment in order to maintain a coherent 
quantum state that is useful for computation, among other applications.
These stray couplings are
known as noise, errors, or {\em decoherence}. Remarkable theoretical results in
{\em fault-tolerant quantum computing} have shown that we can correct these
seemingly continuous errors using discrete operations, unlike classical
analog computation.

Most current approaches to fault-tolerance focus on
quantum error correcting (QEC) codes, which encode
each logical qubit into many physical qubits. However, every QEC code introduces
an overhead of more physical qubits, which increases the probability of
failure. We need to achieve a ``fault-tolerant threshold'' for a single qubit's
error probability in order to ensure we get a net benefit from applying QEC.
Threshold theorems for a variety of architectures have proven that this single
qubit error is anywhere from $10^{-3}$ to $10^{-6}$ \cite{NC2000}.
If $p$ is the single qubit error and $p_{th}$ is the threshold error rate,
the error rate for the encoded logical qubit is $p(p/p_{th})^{2^k}$ after
$k$ levels of concatenation \cite{NC2000}.
Therefore, to fully realize the benefits of driving down error rates with
concatenated QEC codes, we need the ratio $p/p_{th}$ to be as small as possible,
and therefore our goals for $p$ are much more stringent than for unencoded
qubits.

\section{Dynamical Decoupling}

Dynamical decoupling is a promising approach for achieving this much lower error
rate without introducing more qubits or measurement---only precise, unitary quantum
control operations are needed via pulses of energy in a specific sequence, with
certain durations and delays between them. (In the case of trapped ions, these
are pulses of optical laser power).
Based on open-loop control techniques pioneered with
nuclear magnetic resonance (NMR), dynamical decoupling uses sequences of
precisely timed pulses to perform quantum bit- or phase-flips, coherently
averaging out fluctuations from the environment \cite{VL1998}.

Dr. Biercuk was able to use a technique related to the famous Hahn spin-echo
sequence \cite{Vandersypen2004}, with multiple evenly-spaced pulses,
to increase phase coherence times in trapped beryllium ions from 2.5 milliseconds
to 700 milliseconds, an
increase of {\em two orders of magnitude} \cite{BUVSIB2009a}.
(Coherence times are inversely related to decoherence-induced error rates,
where increasing the first is equivalent to reducing the second.)
Trapped ions have an advantage over
other qubit candidates in possessing relatively long coherence times,
identical energy characteristics from qubit to qubit, and a recent
demonstration of a fully-programmable two-qubit trapped ion quantum processor
 \cite{Hanneke2009}.

Unfortunately, the NMR community has long known about several limitations
to a simple DD approach, for example a repeated sequence of
pulses attributed to Carr, Purcell, Meiboom, and Gill (CPMG). The CPMG
sequence is only effective against slowly varying phase noise (relative to
the timescale of the experiment), whereas
in a general quantum information setting, we may experience phase noise
of arbitrary spectral density.
Recent theoretical results indicate that more general kinds of noise can be further
suppressed with longer and more complicated pulse sequences called
Uhrig Dynamical Decoupling (UDD) \cite{Uhrig2007}, where the delays between various
pulses in a sequence can be varied precisely.

Until recently, experimental errors were too great to allow the precise
implementation of these new sequences. However, Dr. Biercuk's group at NIST
were able to overcome these limitations by achieving high operational
and measurement fidelities and engineering the noise
environment itself \cite{BUVSIB2009b}, and thus were able to achieve
the first ever experimental demonstration of UDD. This confirmed the
possibility of
suppressing of errors by 8 to 10 orders of magnitude for some kinds of
high-frequency noise.

Dynamical decoupling in its original form can thought of as a ``refreshing'' process for arbitrary
quantum states
in a quantum memory \cite{BUVSIB2009c}. This provides much needed volatile storage in between
computational operations, similar to a DRAM for classical computers. In the
future, this research could give rise to a ``quantum firmware'' which will
provide a platform for higher-level quantum computation.
However, storage is not enough for computation. How can
we actually transform these states into other states, in order to compute
useful functions?

A further theoretical framework has been proposed to extend DD to 
perform non-trivial logic functions (gates),
called Dynamical Controlled Gates (DCGs)
\cite{Khodjasteh2009}.
This is crucial for performing real quantum computations, as small errors in
enacting individual gates may accumulate and render
the overall computation useless, unless additional suppression is done.
Therefore, even though DD has already demonstrated great promise, more work
needs to be done to fully exploit its potential.

\section{Project Description: Pulse Programming}

While much of the previous discussion has been abstract, the actual work of
this project will be very concrete: the construction of an electronic
device called a pulse programmer to generate electrical pulse sequences
for dynamical decoupling,
and the implementation and testing of software to control it. The pulses
are electrical signals, either analog alternating-current (AC) at some
radio frequency (around 100-400 MHz) or digital signals corresponding to
standard transistor-to-transistor (TTL) logic levels (0 to 5 volts) to control
and synchronize other experimental equipment.
The pulses in DD sequences are generally at radio frequencies (RF) 
and need precise and time-varying
frequency, phase, and amplitude control, with low noise and in several
independent channels. This can be achieved using modern direct-digital-synthesis
(DDS) chips which approximate an analog waveform using digital sampling
techniques. In addition, quickly changing DDS output is needed to create
real-time optimized sequences for a given noise environment, thereby
improving the results in \cite{BUVSIB2009c}.

In addition, the system will also need to generate arbitrary analog waveforms using
a digital-to-analog converter (DAC).
Finally, the system will need to be able to count input pulses from a
photomultiplier tube (PMT) and perform conditional actions based on these
counts, such as branching on whether a threshold has been reached in order to
perform state detection.
Such a
device is a commonplace need for all quantum computing technologies,
including trapped ions,
NMR, superconducting qubits, quantum dots, and others yet to discovered
\cite{Pham2005}. Only specific
features and parameters vary from technology to technology and from experiment
to experiment. Therefore, it is possible to develop a set of
general-purpose, reusable components (both hardware and software)
which can be combined to form a pulse
programmer which is tailored to a particular laboratory's requirements
\cite{Petersen2007} \cite{Schindler2008}.

Because Dr. Biercuk is starting a new lab, he will have the
opportunity to redesign and improve the pulse programming system he
used at NIST
to validate the theoretical proposals above and further extend his
experimental results. For example, the time resolution of the pulse programmer
determines how fast a signal can change; smaller time resolutions allow
pulse sequences to have more accurate delays and therefore greater fidelity
in performing quantum control operations. The data given in \cite{BUVSIB2009c}
was taken at a 50 nanosecond time resolution, while the proposed system in this project
will be
able to achieve 10 nanosecond time resolution. Furthermore, with the new system,
he will be able to optimize pulse sequences in real-time using
faster hardware feedback.

The proposed project will include designing, assembling, and testing
custom circuit boards, integrating them with commercial boards, and
writing control software to perform the above tasks.
This system will be constructed from both
open source and commodity off-the-shelf components, some of which have been
tested in other ion trap experiments, as described in \cite{pp2005},
in order to decrease
its cost and increase its reusability.
Furthermore, the entire design and construction process will be documented
on a public website \cite{pp2005} as a resource to the trapped ion
quantum computing community.
Thus, this project will form one part of a general-purpose quantum control toolkit,
which is the larger goal of Dr. Biercuk's laboratory.

\section{Applicant}

I have extensive past experience in
designing and implementing electronic pulse programming systems to control
ion traps \cite{Pham2005} \cite{Petersen2007} \cite{Schindler2008}.
In addition to openly contributing
to the research community for over five years, I have personally
engineering systems in three different laboratories in Innsbruck, Austria
\cite{Riebe2006}; Garching, Germany; and Seattle, Washington, in the U.S.
I have also maintained an open source project on the internet
\cite{pp2005}
from which other
researchers can learn and to which they have contributed. I have
been cited by the company MagiQ in their successful
Small Business Innovation Research (SBIR)
grant for pulse programming systems \cite{MagiQ2007}
and have also been retained as a consultant for the same company.

However, what I currently lack, and what I hope to gain from this fellowship,
is a deeper understanding of atomic physics and the ability to help propose
and design new experiments. Dr. Biercuk would make an ideal mentor in this
regard, so that in the future I can make a more intellectual contribution to
the field in addition to the technical skills I will be bringing to this
collaboration.

I
frequently correspond with professors, post-docs, and grad students from groups
in Germany, Denmark, England, Austria, and across the U.S. in my quest to
better understand the scientific process. It is my belief that we are limited
more by social and political constraints rather than technological or physical
limitations. My work so far in this field is part of a
new trend of applying reconfigurable computing and open source
methodologies to physics research, as evidenced by NIST Boulder's decision
to release their ion trap control software as open source \cite{ionizer2010}.
I hope to bring this approach to the University of Sydney in Australia,
and by extension other universities in East Asia and the Pacific Rim. In
unifying their technical approaches to common problems, I hope this will open
the door to future collaborations and a greater sense of community between
the EAPSI region and the rest of the world. In the process, I hope my work
with Dr. Biercuk will bring us closer to a working quantum computer.

\bibliography{ppham-project-description}
\bibliographystyle{tocplain}

\end{document}