\documentclass{article}

\usepackage{eprint}

\setlength{\textheight}{9.5in}	% more margin hacking.
\addtolength{\textwidth}{1in}	% same here....
\addtolength{\topmargin}{-0.75in}	% repairing LaTeX's huge margins...
\addtolength{\hoffset}{-0.75in}	% repairing LaTeX's huge margins...
\addtolength{\marginparwidth}{-.2in}	% The word ``Memberships'' is really long below

\begin{document}

\thispagestyle{empty}           % no headers on the first page

\begin{center}
\LARGE{2010 NSF EAPSI Project Synopsis and Timeline}\\
\Large{Paul Pham}
\end{center}

\section{Synopsis}

My proposed project is to construct an electronic pulse programmer to
provide high-fidelity quantum control for trapped ion experiments in
the laboratory of Dr. Michael Biercuk. In particular, the pulse programmer will
operate
at 100 MHz (corresponding to a 10 ns time resolution) and have modules
for radiofrequency analog generation, arbitrary analog waveforms, and
input counting.

The project will be documented in real-time on the website
http://pulse-programmer.org as a resource to the larger ion trap community.
Each week of the timeline
below will include documentation of that week's work with a blog update and
uploaded photos.
All related designs, software, and documentation will be released as
open source.

The circuit boards, electronic components, chassis, and all materials and
supplies will be paid for out of Dr. Biercuk's funding, independently of the
EAPSI fellowship.

\section{Timeline}
\begin{itemize}
\item Week 0: Before arriving in Australia

Discuss logistics of commercial boards to order and design parameters for
custom boards to design.
Begin custom board design, choose components.

\item Week 1: June 14-20, 2010

Set up laboratory workbench, test all boards and components brought from the U.S.
Finish custom board designs, send out for fabrication.
Design system and wiring scheme for controlling all board from FPGA sequencer,
including clock system.
Physically lay out rackmount chassis to store entire pulse programmer,
including faceplate design.

\item Week 2: June 21-27, 2010

Order parts and chassis.
Write software/firmware to control the boards from the FPGA sequencer.

\item Week 3: June 28 - July 4, 2010

Receive custom boards back from fabrication.
Receive parts and chassis.
Construct rackmount chassis and faceplate.
Test physical layout in rackmount chassis.
Assemble custom boards (or outsource).

\item Week 4: July 5 - July 11, 2010

Wire together existing boards, screw into chassis.
Mid-project checkpoint.

\item Week 5: July 12 - 18, 2010

Receive back assembled custom boards. Add to chassis.
Test generating TTL and arbitrary analog waveforms.

\item Week 6: July 19 - July 25, 2010

Test generating RF waveforms using DDS with specified phase, frequency, and
amplitude.

\item Week 7: July 26 - August 1, 2010

Test PMT input counting.

\item Week 8: August 2 - August 8, 2010

Integrate into laboratory setup.
Hand over keys / control codes.

\end{itemize}

\end{document}