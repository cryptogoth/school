\documentclass[12pt]{article}
\usepackage{scribe}
\usepackage[all]{xy}
\Scribe{Paul Pham}
\Lecturer{Venkatesan Guruswami}
\LectureNumber{10}
\LectureDate{1 November 2006}
\LectureTitle{Justesen Codes and Reed-Solomon Decoding}
\begin{document}
\MakeScribeTop

In the last lecture, we introduced the first binary code with
asymptotically
positive rate and distance as well as an explicit polynomial-time
construction. We give that construction in this lecture. In the
process, we introduce an efficient decoder for Reed-Solomon codes.
But first, we recall the Zyablov,
Gilbert-Varshamov, and MRRW (record) bounds below:

\begin{eqnarray}
R_{Zyablov}(\delta - \frac{1}{2} - \epsilon) & = & \Omega(\epsilon^3) \\
\delta(R) & = & \max_{R < r < 1} (1 - \frac{R}{r})H^{-1}(1-r)\\
R_{GV}(\frac{1}{2} - \epsilon) & = & \Omega(\epsilon^2)\\
R_{MRRW}(\frac{1}{2} - \epsilon) & = & 2\epsilon^2 \log{\frac{1}{\epsilon}}
\end{eqnarray}

%------------------------------------------------------------------------------
\section{Justesen Code Construction}

Up until now, we have seen bounds that tell us good codes should be
possible but we didn't know how to construct such codes explicitly.
The binary Reed-Solomon code gave us good rate but poor distance.
Intuitively, we should be able to concatenate it with an inner code that
would ``amplify'' its distance. Enumerating and searching codes takes
$O(2^n)$ time, so we can't just find one good inner code and use it for
all outer blocks.
A key insight of the Justesen code is the use of different inner codes
from an easy-to-find ensemble, most of which have good distance.

%By a theorem due to Wozencraft, we know there are infinite families of
%codes ensembles, which are infinite

The Justesen code concatenates Reed-Solomon on the outside
with an ensemble of codes over a field of $2^m$
elements on the inside.
We denote $n$ as the number of non-zero field elements, $2^m -1$,
and therefore each field element can be encoded into binary
with $m = O(\log n)$ bits. The associated Reed-Solomon polynomial
is $mk$-bits long:

\begin{equation}
f(x) = z_0 + z_1x + \ldots + z_{k-1}x^{k-1}
\end{equation}

We then evaluate $f(x)$, as well as a correlated polynomial $xf(x)$
at all non-zero field elements ($\{\alpha_i\}$). This gives us $2n$ field
elements, $(f(\alpha_0), \alpha_0f(\alpha_0), \ldots, f(\alpha_{n-1}), \alpha_{n-1}f(\alpha_{n-1})$, which can be viewed as a $2nm$-bit string.

\begin{displaymath}
\xymatrix{
f(x) \ar[rr]^{C_{out} = R-S} & & *+[F]{f(\alpha_0)} \ar[d]^{C_{in}^1} & \ldots & *+[F]{f(\alpha_{n-1})} \ar[d]^{C_{in}^{n-1}} \\
 & & *+[F]{(x,\alpha_0x)} & \ldots & *+[F]{(x,\alpha_{n-1}x)} }
\end{displaymath}

In the above figure, the inner codes functions are
\begin{equation}
C_{in}^i(x) = (x, \alpha_{i}x
\end{equation}

Note that these functions are defined over
$\mathbb{F}_{2^m} \rightarrow \mathbb{F}_{2^m} \times \mathbb{F}_{2^m}$.
However, we can easily convert this to a binary function by embedding
each field element as an $m$-bit vector while preserving linearity,
as before: $\sigma: \mathbb{F}_{2^m} \rightarrow \mathbb{F}_{2}^m$
where $\sigma(x+y) = \sigma(x)+\sigma(y)$. Now we can define our
$C_{in}^i$ functions over $\mathbb{F}_{2}^m \rightarrow \mathbb{F}_2^{2m}$.
And of course, we should use binary Reed-Solomon on the outside.

Now we will argue that the rate and distance for this concatenated
binary code are asymptotically good.

%------------------------------------------------------------------------------
\section{Justesen Code Rate and Distance}

First, the rate: $R = \frac{k}{2n}$, or in other words, half the rate of
Reed-Solomon. This makes sense because we are transmitting the same message
with twice as many symbols.
Alternatively, note that each $C_in^{i}$ defined
above is a rate $\frac{1}{2}$ linear code.

If all but an $\epsilon$-fraction of inner codes have distance $d$, then
the concatenated code has distance $\ge (D-\epsilon n)d$.
Now we need to show that greater than a constant fraction of the
inner ensemble has good distance.

\begin{fact}[
Most inner codes $C_{in,i}$ have distance
$\delta > H^{-1}(\frac{1}{2})-\epsilon$.
]
\textit{Proof:}
Let $y$ be a non-zero element of $\mathbb{F}_2^{2m}$ which we can write
as a tuple of elements $(y_1,y_2)$ where $y_1,y_2 \in \mathbb{F}_2^m$.

If $y$ is in some particular code $C_{in}^i$ then $y_2 = \alpha_{i-1}y_1$
by construction in the previous section. Then this code is fixed
by the ratio $y2/y1$. Note that neither $y_1$ nor $n_2$ can both be zero,
otherwise the whole codeword will be zero contrary to our assumption.
If one is zero and the other is not, then it is not in any code.
Otherwise, if $y$ is not in any code, then we can mark it as an erasure
and throw it out.

From the above argument, each non-zero element $y \in \mathbb{F}_2^{2m}$
is in at most one of the codes $C_{in}^i$.\footnote{In fact, our proof still goes
through as long as each element is in at most some constant number of codes.}
Therefore, we can bound the
number of ``bad'' codes ($\epsilon n$ codes with distance $\le d$)
is bounded above by the number of bad
codewords (which are $2m$ bits long).
%
\begin{eqnarray}
\textrm{\# codes with dist } < d & \le & Vol(0, d\cdot 2m)\\
           & \le & 2^{H(d)2m} \textrm{ by Lemma 3.3 in Lecture 3 }\\
           & = & 2^{(\frac{1}{2} - \epsilon)2m}\\
           & = & \frac{2^m}{2^{\epsilon\cdot 2m}}\\
           & < & \epsilon(2^m-1)\\
           & = & \epsilon n
\end{eqnarray}
This decreases 
$\square$
\end{fact}
%
Therefore, the Justesen code has the distance below:
%
\begin{eqnarray}
\textrm{distance} & \ge &((1-R)n - \epsilon n)d\cdot 2m \\
                  & = & (1-R-\epsilon)nH^{-1}(\frac{1}{2}-\epsilon)2m\\
\delta            & \ge & (1 - R -\epsilon)H^{-1}(\frac{1}{2} - \epsilon)
\end{eqnarray}
%
The last equation above is equivalent to the Zyablov bound given at
the beginning of these notes.
In conclusion, the code can be used for any rate $< \frac{1}{2}$ and
still be on the Zyablov bound.

%------------------------------------------------------------------------------
\section{Digression: Reed-Solomon $\diamond$ Hadamard}

Leaving behind Justesen codes for a moment, we will examine a construction
with some interesting properties.
We concatenate
Reed-Solomon as the outer code with Hadamard as the inner code,
where $\mathrm{Had}[a](x) = ( a\cdot x ), \forall x \in \mathbb{F}_{2m}$.
We call the outer rate $\epsilon$ and
take the outer distance to be $\delta_{out} = 1 - \epsilon$ and the
inner distance to be $\delta_{in} = \frac{1}{2}$.

\begin{displaymath}
\xymatrix{
f(x) \ar[r] & & *+[F]{f(\alpha_0)} \ar[dl] \ar[dr] & & \ldots & & *+[F]{f(\alpha_{n-1})} \ar[dl] \ar[dr] & \\
               & | & 2^m \textrm{ bits } & | & \ldots & | & 2^m \textrm{ bits } & | }
\end{displaymath}

The concatenated distance is $\delta = \frac{1}{2}(1-\epsilon)$, which
implies that all non-zero codewords have relative weight within the range
$[\frac{1}{2} - \epsilon, \frac{1}{2}]$, otherwise known as an
$\epsilon$-biased space.

Alon, Goldreich, H\aa stad, and Peralta \cite{aghp92}
showed that $k$ independent
variables (a message) could be used to simulate a ``small'' distribution over
$n \gg k$ variables (codewords)
within $(k/\epsilon)^2$ vectors, a polynomial in $k$.
In error-correcting terms, the generator matrix for this code is a
$(k/\epsilon)^2 \times k$ matrix but is equivalent
to a code of size $2^{2m}$ with respect to any linearity test (dot product)
for a large fraction of the inputs.

If we characterize the outer Reed-Solomon code as $[n,k,n-k]_n$ and the
inner Hadamard code as $[n,\log n, \frac{n}{2}]_2$, we get a
concatenated code of $[n^2,k\log{n},\frac{n}{2}(n-k)]$.
The concatenated relative distance can be expressed as
$\delta = \frac{1}{2} - \frac{k}{n}$.
An interesting fact is that we can choose the outer rate $k/n$ to get
a constant total distance but poor rate, starting with an outer code that
gave us constant rate at the expense of poor distance.
However, the inner code has both poor rate and distance.
Intuitively, it seems that if we concatenate an inner code with good distance
and the Reed-Solomon code with good rate, we should be able to get
an asymptotically good code.\footnote{These insightful remarks came from Madhu Sudan's MIT course notes for 6.897 in Fall 2001.}

Now back to our regularly-scheduled programming.

%------------------------------------------------------------------------------
\section{General Linear Erasure Decoding}

Recall that an erasure is an error that we know with certainty,
whereas in general we may not be able to detect all errors.
From the first lecture, we know that in a code with distance $d$, any
received message with $(d-1)$ erasures or fewer is consistent with at most one
codeword. Therefore, we can ``fill in'' indeterminate symbols up to $(d-1)$
erasures.

How can we do this efficiently? It's always easy for a linear code if we
are given the generator matrix $G$. If the message is $x \in \mathbb{F}_q^k$,
compute and send $Gx \in \mathbb{F}_q^n$.
Suppose that some symbols may be corrupted
in transit, and the receiver gets $y \in \mathbb{F}_q^n$. 
Call the subset of \emph{uncorrupted}
positions $S \subseteq \{1,\ldots,n\}$, which we know by assumption.
Then we merely have to solve a linear system, where subscript $y_S$ means
$y$ restricted to the positions in $S$ and likewise for $Gx_s$.

\begin{equation}
Gx_S = y_S
\end{equation}

Without loss of generality, we can reorder the rows of $G$ and $y$ so that
the positions in $S$ are at the top and the corrupted positions are at the
bottom. Then we simply solve the linear system restricted to the top
$|S|$ rows of $G$ and $y$.
%
\begin{equation}
\left[ \begin{array}{ccc}
 &  &  \\
 & G_S &  \\
 & & \\
\hline
& G_{\overline{S}} &
\end{array} \right]
\left[ \begin{array}{c}
\\
x \\
\\
\end{array} \right]
=
\left[ \begin{array}{c}
 \\
y_S \\
 \\
\hline
y_{\overline{S}}
\end{array} \right]
\end{equation}
%
If there are few enough erasures ($|S| \ge k$), we will get a unique solution.
If there are more erasures, we can only output a linear space of solutions,
which is a form of list-decoding.

%------------------------------------------------------------------------------
\section{Reed-Solomon Decoding}

The last piece in our code is the decoding algorithm, which itself
depends on being able to decode Reed-Solomon efficiently.
First, we'll summarize the work leading up to the R-S decoder
we know and love today. Then we'll describe R-S erasures-only decoding as
a subroutine to the final R-S errors-and-erasures decoder in the final
section.

%----------------------------------------------------------------------------
\subsection{History}

The history Reed-Solomon decoding can be traced back to the first efficient
algorithm for binary BCH codes by Peterson in 1960 \cite{peterson60}, since
R-S codes are a special case of BCH codes.
This algorithm was able to correct errors up to half the distance
($e \ge \lfloor \frac{d-1}{2} \rfloor$) in $O(n^3)$ time using
error locator polynomials.
Interestingly, the original Reed-Solomon paper appeared earlier that year,
when digital technology was not advanced enough to implement the decoding
procedure.

For Reed-Solomon codes, the first practical decoder is attributed to
Berlekamp in 1968 \cite{berlekamp68} which solved the key equation
iteratively. It was reformulated in a
shift register setting by Massey in 1969 \cite{massey69} which was easy to
implement. This procedure, often called the Berlekamp-Massey algorithm, is
used in all commercial CD and DVD players, transmissions from the
Voyager satellite, and many other applications in mass storage and
communication. The corresponding patent by Berlekamp and Welch in
1986 \cite{bw86} was a key asset of Berlekamp's company Cyclotomics.
A succinct description can be found in the appendix of \cite{gs92} by
Gemmell and Sudan in 1992.
This algorithm runs in $O(n^2)$ time and will be presented in the next section.

An alternate $O(n^2)$ algorithm using Euclid's GCD method to solve the
key equation was discovered by Sugiyama, Kasahara, Hirasawa, and Namekawa
in 1975 \cite{skhn75}. This was generalized to a single algorithm for both
errors and erasures in \cite{skhn76} a year later, and in the same
journal issue was published Justesen's algorithm \cite{j76} with the
best known upper bound to date, $O(n\log^2 n)$.

%------------------------------------------------------------------------------
\subsection{Reed-Solomon Erasure Decoding}

Similar to the process for general linear codes above,
erasure decoding for Reed-Solomon codes is straightforward.
Each code has an associated polynomial $f(x)$ for $x \in \mathbb{F}_q^k$.
Suppose the sender transmits $(f(\alpha_0),\ldots,f(\alpha_{n-1}))$, and
again we know the uncorrupted positions $f(\alpha_i)$ for $i \in S$ with
$|S| \ge n - (d-1) = k$.

Since $\mathrm{deg}(f(x)) \le k-1$, we can recover $f$ using $k$ or more
distinct points using polynomial interpolation according to the 
unisolvence theorem. This can be done using
FFT in $O(n\log n)$ time or with Lagrange interpolation in $O(n^2)$ time.
The latter method computes $n$ basis polynomials $p_j$ for $0 \le j < k$.
%
\begin{equation}
p_j(x) = \prod_{i=0,i\ne j}^k \frac{x-\alpha_i}{\alpha_j - \alpha_i}
\end{equation}
%
The interpolated polynomial is then:
%
\begin{equation}
f(x) = \sum_{i =0}^k f(\alpha_i)p_i(x)
\end{equation}
%
Now we must deal with the case of some of the received symbols
$f(\alpha_i)$ being in error.

%------------------------------------------------------------------------------
\pagebreak
\subsection{Reed-Solomon Error Decoding Problem}

As above, our sender's encoded message was
$(f(\alpha_0),\ldots,f(\alpha_{n-1})) \in \mathbb{F_q^n}$ and the
receiver gets $y = (y_0, y_1, \ldots, y_{n-1}) \in \mathbb{F}_q^n$.
If a symbol in position $i$ was received correctly (without error),
then $y_i = f(\alpha_i)$.
Let's say that a fraction $e$ of the received symbols have been
corrupted, that is $y_i \ne f(\alpha_i)$. We assume
$2e < d = n-k+1$, otherwise we have no chance for uniquely correcting the
error. If we can
figure out which positions are in error, we can throw them out and just
do erasure decoding as in the previous section.

More formally, our problem can be stated as follows. Given $n$ pairs of
$(\alpha_i, y_i) \in \mathbb{F}_q, i \in \{0,1,\ldots,n-1\}$, determine if
there is a polynomial $f(x) \in \mathbb{F}_q[x]$ of degree $\le k-1$
such that $f(\alpha_i) \ne y_i$ for at most a fraction $e$ of the $n$ pairs.
If there is, find it. In our case, the points $\alpha_i$ must be distinct
for us to apply polynomial interpolation, but the problem still makes sense
without this condition.

%----------------------------------------------------------------------------
\subsection{Reed-Solomon Error Decoding Algorithm}

Recall that if there is no error in position $i$, then $y_i = f(\alpha_i)$.
We define an error locator polynomial, which we don't know \emph{a priori}.
%
\begin{equation}
E(x) \triangleq \prod_{f(\alpha_i) \ne y_i} x - \alpha_i
\end{equation}
%
It is easy to verify that for $0 \le i < n$,
$E(\alpha_i)y_i = E(\alpha_i)f(\alpha_i)$. Define the key equation:
%
\begin{equation}
N(x) \triangleq E(x)f(x)
\end{equation}
%
Then for $0 \le i < n$, we have the linear system of equations
$N(\alpha_i) - E(\alpha_i)y_i = 0$. Note that $E(x)$ has degree $\le e$
and $f(\alpha_i)$ has degree $\le k-1$.
 Call one particular solution
of this system $x$ and define the two polynomials below:
%
\begin{eqnarray}
E_1(x) & = & a_0 + a_1x + \ldots + a_e x^e\\
N_1(x) & = & b_0 + b_1x + \ldots + b_{e+k-1} x^{e+k-1}
\end{eqnarray}
%
This is a homogenous linear system in the unknowns
$a_i$ for $0 \le i \le e$ and $b_j$ $0 \le j < e+k$.
There may be other solutions $x$, but the ratio between the two
polynomials will be the same.
%
\begin{enumerate}
\item
If $E_1(x)$ does not divide $N_1(x)$, output ``too many errors.''
Otherwise, define $f(x) = N_1(x) / E_1(x)$.
\item
If $\Delta(RS(f(x)), y) > e$, output ``too many errors'' otherwise
return $f(x)$.
\end{enumerate}
%
We know that a non-zero solution $x$ must exist by a root/degree
counting argument. Define another polynomial:
%
\begin{equation}
R(x) = N_1(x) - f(x)E_1(x) = 0
\end{equation}
%
The degree of $R(x)$ is $\le e+k-1$. Assume that $\le e$ errors happen.
For the positions with no errors, $y_i = f(\alpha_i)$ and we have:
%
\begin{eqnarray}
R(\alpha_i) & = & N_1(\alpha_i) - f(\alpha_i)E_1(\alpha_i)\\
            & = & N_1(\alpha_i) - y_i E_1(\alpha_i)
\end{eqnarray}
%
$R(\alpha_i) = 0$ for at least $n-e$ values of $i$.
If the number of roots $n-e$ is greater than the degree, $R(x)$ is the
zero polynomial, which implies a non-zero solution because all the
$\alpha_i$'s are non-zero. Rearranging the inequality, we get:
%
\begin{equation}
2e < n-k+1 = D
\end{equation}
%
where $D$ is the distance of Reed-Solomon.
This shows that we can correct errors and erasures up to half
the distance. $\square$

\begin{thebibliography}{99}

\bibitem{peterson60} W.W.~Peterson.
``Encoding and error-correction procedures for the Bose-Chaudhuri codes.''
\emph{IRE Transactions on Information Theory}, Vol. 6, pp. 459-470, 1960.

\bibitem{berlekamp68} E.R.~Berlekamp.
\emph{Algebraic Coding Theory}. McGraw-Hill, New York, 1968.

\bibitem{massey69} J.L.~Massey.
``Shift register synthesis and BCH decoding.''
\emph{IEEE Transactions on Information Theory}, Vol. IT-17, pp. 122-127,
1969.

\bibitem{skhn75} Y.~Sugiyama, M.~Kasahara, S.~Hirasawa, T.~Namekawa.
``A  method for solving key equation for decoding Goppa codes.''
\emph{Information and Control} Vol. 27, pp. 87-89, 1975.

\bibitem{skhn76} Y.~Sugiyama, M.~Kasahara, S.~Hirasawa, T.~Namekawa.
``An erasures-and-errors decoding algorithm for Goppa codes.''
\emph{IEEE Transactions on Information Theory}, Vol. 22, No. 2, March 1976.

\bibitem{j76} J.~Justesen.
``On the complexity of decoding Reed-Solomon codes.''
\emph{IEEE Transactions on Information Theory}, Vol. 22, No. 2, March 1976.

\bibitem{bw86} E.R.~Berlekamp and L.R.~Welch.
``Error Correction of Algebraic Block Codes.'' U.S. Patent No. 4,633,470.
30 December 1986.

\bibitem{gs92} P.~Gemmell and M.~Sudan.
``Highly-Resilient Correctors for Polynomials.''
\emph{Information Processing Letters}, 1992.\\
\texttt{http://www.eecs.berkeley.edu/Pubs/TechRpts/1992/CSD-92-661.pdf}

\bibitem{aghp92} N.~Alon, O.~Goldreich, J.~H�stad, and R.~Peralta.
``Simple constructions of almost k-wise independent random variables.''
\emph{Random Structures and Algorithms}, Vol. 3, pp. 289-304, 1992.
\end{thebibliography}

\end{document}
